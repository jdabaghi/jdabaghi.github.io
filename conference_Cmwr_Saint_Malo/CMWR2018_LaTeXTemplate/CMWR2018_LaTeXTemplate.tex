\documentclass{cmwr2018}[10]

% Set the title of your contribution
\title{A posteriori error estimates, stopping criteria, and adaptivity for a two phase flow with exchange between phases as a nonlinear complementarity problem}

% Set the authors and affiliations
\author{Ben Gharbia Ibtihel, (ibtihel.ben-gharbia@ifpen.fr), IFPEN, France \\
Dabaghi Jad, (jad.dabaghi@inria.fr), Inria Paris, France\\
Vincent Martin, (vincent.martin@utc.fr), UTC Compiègne, France\\
Martin Vohral\'ik, (martin.vohralik@inria.fr), Inria Paris, France}

% Set three keywords
\keywords{complementarity condition, semismooth Newton method, a posteriori error estimate, adaptivity and stopping criterion}

\begin{document}
\maketitle

%\section*{Introduction}

\begin{abstract}
\label{ref:abstract}
We develop an a posteriori-steered algorithm for the two-phase compositional flow with exchange of components between the phases in porous media. The discretization relies on the backward Euler scheme in time and the finite volume scheme in space. The resulting nonlinear system is solved via an inexact semismooth Newton method treating the phase transition. Numerical experiments are given for the semismooth Newton-min algorithm and the GMRES solver, showing good quality of the  estimates and of the adaptive stopping criteria.
\end{abstract}


%Authors are expected to submit
%\begin{enumerate}
%	\item \textbf{\textcolor{red}{A short abstract of maximum 80 words}} loaded directly on the submission site. 
%	\item \textbf{\textcolor{red}{A regular abstract of maximum 2 pages (20 Mo)}} written following the format of this template. The file has to be translated into \textbf{\textcolor{red}{PDF}} before electronic submission following the instructions given on the submission site. 
%\end{enumerate}
%
%\section*{Format Specifications of regular abstract}
%
%\paragraph{General:}
%The paper must be written in English on DIN A4 size paper (210 mm x 297 mm) with margins of 2 cm. The abstract including figures, tables and references must be at most 2 pages. 
%
%\paragraph{Text:}
%The normal text should be written single-spaced, justified, using 10pt (Times New Roman). 
%
%\paragraph{Title, Authors and Affiliations}
%should be centered above the text.
%
%\paragraph{Headings}
%should be written left aligned and boldface, with upper and lower case letters.
%
%\paragraph{Page number:}
%No page number is to be given.
%
%\paragraph{Tables and Figures}
%All tables and figures should be numbered consecutively and captioned. They should be included in the text close to the place where they are referred, such as Table \ref{tab:table} and Figure \ref{fig:figure}. The caption title should be written centered. 
%
%\begin{table}[hbt]
%\begin{center}
%\begin{tabular}{*{3}{|c}|}
%\hline
%~ & ~ & ~ \\
%\hline
%~ & ~ & ~ \\
%\hline
%\end{tabular}
%\end{center}
%\caption{Example of a table}
%\label{tab:table}
%\end{table}
%
%
\begin{figure}[hbt]
\centering
\includegraphics[height=5cm]{Logo_CMWR2018}
\caption{Example of a figure}
\label{fig:figure}
\end{figure}
%
%
%\paragraph{Format of References}
%References should be quoted in the text by arabic numbers in square brackets,  
%\cite{Zienkiewicz,Idelsohn} 
%and  grouped together at the end of the Abstract in numerical order as shown in these instructions.
%
%{\small
%\begin{thebibliography}{9}
%\bibitem{Zienkiewicz} O.C. Zienkiewicz and R.L. Taylor. \textit{The finite element method}, McGraw Hill, Vol. I.,
%(1989), Vol. II., (1991).
%\bibitem{Idelsohn} S. Idelsohn and E. O\~nate. Finite element and finite volumes. Two good friends. \textit{Int. J.
%Num. Meth. Engng.}, \textbf{37}, 3323--3341, (1994).
%\end{thebibliography}
%}

\end{document}
