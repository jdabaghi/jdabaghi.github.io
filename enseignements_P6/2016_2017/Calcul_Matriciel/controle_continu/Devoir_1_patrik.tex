\documentclass[12pt]{article}
\usepackage[latin1]{inputenc}
\usepackage[T1]{fontenc}
\usepackage[%
      a4paper,%
      textwidth=18cm,%
      top=2cm,%
      bottom=2cm,%
      headheight=25pt,%
      headsep=12pt,%
      footskip=25pt]{geometry}%
\usepackage[frenchb]{babel}
\parindent0pt
\usepackage{longtable}
\usepackage{amsmath, amsfonts}
\usepackage{enumitem}

\renewcommand{\geq}{\geqslant}
\renewcommand{\leq}{\leqslant}
\def\N{\textrm{I\kern-0.21emN}}
\def\C{\mathbb{C}}
\def\K{\mathbb{K}}
\def\R{\textrm{I\kern-0.21emR}}
\def\Q{\textrm{l\kern-0.5emQ}}
\def\Z{\mathbb{Z}}
\newcommand{\ds}{\displaystyle}

\pagestyle{empty}

\begin{document}

{\Large UPMC \hfill 1M004 - Calcul matriciel \hfill 2016-2017}

\vskip -3mm
\noindent \textbf{\hrulefill}

\vskip 1mm

\center{
{\Large  \textbf{TD Section 21.3 -- Contr�le du 6 mars}} \linebreak
{\large \textbf{Dur�e : 45 minutes / Sans documents ni mat�riel �lectronique}}
}

\vskip -2mm
\noindent \textbf{\hrulefill}



%\vskip 2mm

%\centerline{\textbf{Tout appareil �lectronique (calculatrices, t�l�phones portables, etc.) est interdit}}

%\vskip 2mm
{\LARGE
\begin{center}
%\begin{tabular}{|p{7cm}|p{1cm}|p{1cm}|p{1cm}|p{1cm}|p{1cm}|p{1cm}|p{1cm}|}
%\hline
%\textbf{Num�ro d'�tudiant} &     & & & & &   &
%\\
%\hline
%\end{tabular}

\vskip 0.25cm

\begin{tabular}{|p{2cm}|p{6cm}|p{3cm}|p{5cm}|}
\hline
\textbf{Nom} &     & \textbf{Pr�nom   }  &   \\
\hline
\end{tabular}
\end{center}
} %% fin LARGE


{\sl  La pr�sentation et la clart� des raisonnements seront pris en compte dans l'appr�ciation des
copies. Pensez � justifier tous vos r�sultats. }


\vskip 3mm
\hrule

\vskip 3mm
\noindent \textbf{Questions du cours :} \\
\begin{enumerate}[font=\bfseries,label=(Q\arabic*)]
\item  Soit $A$ une matrice carr�e. Donner la d�finition de la trace de $A$.
\item  \emph{Vrai ou Faux:} Soient $A \in \mathcal{M}_{m,n}(\R)$, $B \in \mathcal{M}_{n,p}(\R)$ et $C \in \mathcal{M}_{p,q}(\R)$
\begin{enumerate}
\item Multiplication matricielle est commutative, alors $AB = BA$.
\item La multiplication des matrices est associative, alors $(AB)C = A(BC)$.
\end{enumerate}
%\item  Soit $\alpha \in \R_{+}$. Montrer par r�currence que pour tout $n \in \N$ on a $(1+\alpha)^n \ge 1 + n\alpha$.
%  \item R�soudre le syst�me suivant par la m�thode du pivot de Gauss [1pt]:  %solution (3,2,0)
%\begin{align*} 
%x + y + 2z &=  5 \\ 
%x - y - z &= 1\\
%x + z &= 3
%\end{align*}
%
%\item Donner la d�finition du d�terminant d'une matrice de taille 2  et description de l'inverse d'une matrice de taille 2 [1pt]        
\end{enumerate}

\vskip 0.2cm
  \hrule
\vskip 2mm


 \noindent \textbf{Exercices :}\\
 \begin{enumerate}[font=\bfseries]
 \item Soient les matrices $$A = \begin{pmatrix} 0 & 2 & -1  \\ -2 & -1 & 2 \end{pmatrix}, \quad B = \begin{pmatrix} 1 & 0 & 1  \\ -1 &  1 & -2 \\ 
0 & 2 & 1 \end{pmatrix}, \quad C = \begin{pmatrix} -2 & 1  \\ 1 & 0\\ 
 0 & 2\end{pmatrix} .$$ Le produit $ABC$ est-il d�fini? Si oui, le calculer. M�me question pour le produit $CAB$.
\item Dans $\mathcal{M}_{3,4}(\R)$ soient $A = \begin{pmatrix} -5 & -4 & -3 & -2 \\ -1 & 0 & 1 & 2 \\ 
3 & 4 & 5 & 6\end{pmatrix}$ 
et $B = \begin{pmatrix} -1 & 0 & 1 & 2 \\ 2 & 1 & 0 & -1 \\ 1 & 0 & -1 & 2 \end{pmatrix}$.\\ 
\'Ecrire la matrice ${}^t\!A$ puis calculer  $B\, {}^t\!A$. 
%------------------------------------------------------
 \item R�soudre le probl�me suivant: on utilise $x$ mol�cules de $C_3H_6$ et $y$ mol�cules de $C_2H_4$ pour obtenir 4 mol�cules de $CH_4$. Trouver $x$ et $y$.
%------------------------------------------------------
% \item Soit $x \in \R$. On consid�re la matrice $A_x = \begin{pmatrix}
% 1 & 2 & 3\\
% 1 & 2x &-x^2\\
% 1 & 5 & 6
% \end{pmatrix}$
% \begin{enumerate}
% \item Calculer le d�terminant de la matrice $A_x$.
% \item D�terminer l'ensemble des r�els $x$ tels que $A_x X = B$ a une solution pour tout $B \in \R^3$.
% \item Calculer l'inverse de $A_0$, i.e. en prenant $x = 0$.
% \end{enumerate}
%------------------------------------------------------
 \item En utilisant la m�thode d'elimination de Gauss, r�soudre le syst�me suivant:
$$\left\{\begin{array}{cccccccc}x&+&y&+&2z&=&5\\x&-&y&-&z&=&1\\x&&&+&z&=&3\end{array}\right.$$
 
 \end{enumerate}

\end{document}
