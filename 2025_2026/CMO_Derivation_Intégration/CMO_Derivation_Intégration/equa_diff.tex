
%%%%%%%%%%%%%%%%%%%%%%%%%%%%%%%%%%%%%%%%%%%%%%%
%%%%%%%%%%EQUATIONS DIFFERENTIELLES%%%%%%%%%%%%
%%%%%%%%%%%%%%%%%%%%%%%%%%%%%%%%%%%%%%%%%%%%%%%
\begin{frame}
    \begin{center}
        \Huge{\'{E}quations différentielles linéaires}
    \end{center}
\end{frame}
%%%%
\begin{frame}
  Soit $I$ un intervalle de $\mathbb{R}$ de la forme $I = ]0, T[$ avec $T > 0$.
      Soit $ m \geq 1$ un nombre entier.
    \begin{definition}
      Soit $f : \overline{I} \times \mathbb{R}^m \rightarrow \mathbb{R}^{m}$.
      On appelle équation différentielle ordinaire (EDO) d’ordre $n$, l'équation exprimant la dérivée n-ème $y^{(n)}$ d’une fonction $y : \overline{I} \rightarrow \mathbb{R}^m$, en fonction de ses dérivées d’ordre inférieur $y^{(i)}$, $i = 0,\cdots, n-1$, de la forme
      \begin{equation*}
\forall t \in I, \quad y^{(n)}(t) = f(t, y(t), y^{\prime}(t), \cdots, y^{(n-1)}(t)).
      \end{equation*}
      Une solution de l'EDO est une fonction de $\overline{I} \rightarrow \mathbb{R}^m$, $n$ fois dérivable sur $I$ et telle que l’égalité soit satisfaite pour tout $t \in I$.
    \vspace*{0.2 cm}  \\
\invisible<1>{
    \textcolor{midnightblue}{ 
On appelle problème de Cauchy, la conjonction d’une EDO et d’une donnée initiale,
\begin{equation*}
  \begin{split}
  y^{\prime}(t) &= f(t, y(t))
  \\
  y(t_0) &= y_0
  \end{split}
\end{equation*}
    }
    \invisible<2>{
      }}
    \end{definition}
\end{frame}
%%%%
\begin{frame}
  \frametitle{Exemples de problèmes conduisant à une équation différentielle}
     \begin{enumerate}
      \item
        Charge d'un condensateur à travers une résistance
        \begin{minipage}[b]{0.5\linewidth}
        \begin{figure}
          \centering
\includegraphics[scale = 0.4]{circuit_RC}
        \end{figure}
        \end{minipage}
        \hfill
        \begin{minipage}[b]{0.49\linewidth}
          \begin{itemize}
\item
            $t$ : variable temporelle
          \item       
            $i$ : intensité du courant
          \item
            $q$ : charge du condensateur
            \end{itemize}
         \textcolor{cadmiumgreen}{\textbf{Loi d'Ohm}}
        \begin{equation*}
U = Ri + \dfrac{q}{c}.
        \end{equation*}
          \end{minipage}
        On obtient le problème de Cauchy suivant:
        \begin{equation*}
          \begin{split}
            \dfrac{dq}{dt} + \dfrac{q}{RC}  & = \dfrac{U}{R}
            \\
            q(0) & = 0
          \end{split}
        \end{equation*}
     \end{enumerate}
\end{frame}
%%%%%
\begin{frame}
  \begin{enumerate}
    \setcounter{enumi}{1}
\vspace*{-0.3 cm}
  \item
    Oscillateur harmonique
\vspace*{0.2 cm}
        \\
         \begin{minipage}[b]{0.45\linewidth}
        \begin{figure}
          \centering
\includegraphics[scale = 0.4]{systeme_ressort}
        \end{figure}
        \end{minipage}
         \hfill
   \begin{minipage}[b]{0.54 \linewidth}
     \begin{itemize}
     \item
       \textbf{Système étudié :} point $M$ de masse $m$ accroché à l’extrémité d’un ressort.
       $\ell$ : longueur du ressort à l'instant $t$, $\ell_0$ : longueur au repos.
       Allongement du ressort : $x = \ell - \ell_0$.
     \item
       \textbf{Forces appliquées à $M$ :} poids $\vec{P}$, réaction du support $\vec{R}$, et force de rappel du ressort $\vec{F} = -k (\ell - \ell_0) \vec{e}_x$.
     \item
        \textbf{PFD :} $\dps \sum \vec{F_{\mathrm{ext}}} = m \vec{a} \Rightarrow \vec{P} + \vec{R} + \vec{F} = m \vec{a}$
               \end{itemize}
   \end{minipage}
\vspace{0.1 cm}   \\
   \vspace*{0.1 cm}
        En projetant l'équation précédente sur l'axe $(0x)$ on obtient
        \begin{equation*}
 x^{\prime \prime}(t) + \frac{k}{m}x(t) = 0.
          \end{equation*}
    \end{enumerate}
\end{frame}
%%%%
\begin{frame}
  \frametitle{\'{E}quations différentielles linéaires du premier ordre}
  \begin{definition}
      Soient $a$, $b$, et $c$ trois fonctions continues de $I$ dans $\mathbb{K}$. On considère l'équation différentielle linéaire du premier ordre
      \begin{equation*}
a(t) y^{\prime}(t) + b(t) y(t) = c(t) \qquad \qquad (E)
      \end{equation*}
      On appelle solution sur $I$ de l'équation différentielle toute fonction $f : I \rightarrow \mathbb{K}$ dérivable telle que :
      \begin{equation*}
\forall t \in I, \ a(t) f^{\prime}(t) + b(t) f(t) = c(t).
        \end{equation*}
  \end{definition}
      \textbf{Remarque :}
      Si $c$ est la fonction nulle, l'équation différentielle est dite homogène et on la note en général $(E_0)$.
\end{frame}
%%%%
\begin{frame}
\begin{proposition}
Si $f_0$ est une solution particulière de l'équation $(E)$ et si $\mathcal{S}$ désigne l'ensemble des solutions de l'équation $(E_0)$, l'ensemble des solutions de l'équation $(E)$ est $\left\{ f_0 + g \ | g \in \mathcal{S} \right\}$.
\end{proposition}
\invisible<1>{
  \textbf{Démonstration :}
  \\
\invisible<2>{
  Puisque $f_0$ est une solution particulière de l'équation $(E)$ on a
      \begin{equation*}
a(t) f_0^{\prime}(t) + b(t) f_0(t) = c(t)
      \end{equation*}
      \invisible<3>{
      De même, si $g$ est solution de $(E_0)$ alors on a
      \begin{equation*}
a(t) g^{\prime}(t) + b(t) g(t) = 0.
      \end{equation*}
\invisible<4>{
      En sommant ces deux équations on trouve
      \begin{equation*}
a(t) \left(f_0 + g \right)^{\prime}(t) +  \left(f_0 + g \right)(t) b(t) = c(t)
      \end{equation*}
      ce qui prouve le résultat souhaité.
      \invisible<5>{
}}}}}
\end{frame}
%%%%%
\begin{frame}
  \frametitle{Résolution d'une équation homogène}
    Soient $a, b \in \mathcal{C}^{0}(I, \mathbb{K})$ et on considère l'EDO du 1er ordre
 \textcolor{cadmiumgreen}{${\bf a(t) y^{\prime}(t) + b(t) y(t) = 0}$ ($E_0$)}
\invisible<1>{
 \begin{proposition}
      Si la fonction $a$ ne s'annule pas sur $I$ et si $A$ est une primitive sur $I$ de la fonction $t \mapsto \dfrac{b(t)}{a(t)}$, les solutions sur $I$ de l'équation $(E_0)$ sont les fonctions de la forme :
      \begin{equation*}
t \mapsto \lambda e^{-A(t)} \quad \text{avec} \quad \lambda \in \mathbb{K}
        \end{equation*}
 \end{proposition}
 \invisible<2>{
   \textbf{Démonstration :}
   \\
      Soit $g$ définie sur $I$ par $g(t) = \lambda e^{-A(t)}$ avec $\lambda \in \mathbb{K}$.
      La fonction g est dérivable sur $I$ et
      \begin{equation*}
        \begin{split}
          a(t) g^{\prime}(t) + b(t) g(t) &= a(t) \left( -\lambda A^{\prime}(t) e^{-A(t)} \right) + b(t) \lambda e^{-A(t)}
          =
           \lambda \left(-a(t) \dfrac{b(t)}{a(t)} + b(t) \right)e^{-A(t)}
          \\
          & = 0.
          \end{split}
      \end{equation*}
      \invisible<3>{
        }}}
\end{frame}
%%%%%
\begin{frame}
\frametitle{Résolution de l'équation avec second membre}
Soient $a$, $b$, et $c$ trois fonctions continues sur un intervalle $I$ dans $\mathbb{K}$.
On considère l'équation différentielle :
  \begin{equation*}
a(t) y^{\prime}(t) + b(t) y(t) = c(t) \qquad \qquad (E)
  \end{equation*}
  \invisible<1>{
\textcolor{cadmiumgreen}{Solution générale de $(E)$ :} solution de l'équation homogène  $+$ solution particulière.
    \vspace*{0.4 cm}
    \\
     \invisible<2>{
  \textcolor{red}{Méthodologie pour trouver une solution particulière}
  \begin{enumerate}
  \item
    La nature du second membre donne des indications.
    Par ex, si le second membre est constant on cherchera une solution constante et s'il est trigonométrique on cherchera une combinaison linéaire de fonctions trigonométriques.
\item Si on ne trouve pas facilement une solution particulière on utilise la méthode de la variation de la constante.
  \end{enumerate}
  \invisible<3>{
    }}}
  \end{frame}
%%%%
\begin{frame}
  \frametitle{Méthode de la variation de la constante}
\invisible<1>{
  \textbf{Principe :}
  Cette méthode consiste à rechercher une solution particulière de la forme $y = \lambda y_0$ où $y_0$ est une solution non nulle de $(E_0)$ et $\lambda$ une fonction dérivable sur $I$.
  \invisible<2>{
  \begin{equation*}
    \begin{split}
      \forall t \in I, \ a(t) y^{\prime}(t) + b(t) y(t) & = a(t) (\lambda y_0)^{\prime}(t) + b(t) \lambda(t) y_0(t)
      \\
      & = \lambda(t) \left(a(t) y_0(t) + b(t) y_0(t)  \right) + a(t) \lambda^{\prime}(t) y_0(t)
      \\
      & = a(t) \lambda^{\prime}(t) y_0(t) \ \text{car} \ y_0 \ \text{est une solution de} \ E_0.
      \end{split}
  \end{equation*}
  \invisible<3>{
  Ainsi $\lambda y_0$ est une solution de $(E)$ si et seulement si la fonction $\lambda$ vérifie :
  \begin{equation*}
\forall t \in I, \ \lambda^{\prime}(t) = \frac{c(t)}{a(t) y_0(t)}
  \end{equation*}
  si, et seulement si,
  \begin{equation*}
\lambda(t) = \int \dfrac{c(s)}{a(s) y_0(s)} ds.
  \end{equation*}
  \invisible<4>{
    }}}}
  \end{frame}
%%%
\begin{frame}
\frametitle{Exercice}
Résoudre sur $]0, + \infty[$ l'équation $(E)$: $t y^{\prime}(t) - y(t) = t^2 e^t$.
      \vspace*{0.2 cm}
      \\
      \corrige{
        \begin{enumerate}
          \invisible<1>{
          \item
    \textbf{Résolution de l'équation homogène} $(E_0)$ associée à $(E)$.
\begin{equation*}
y_0(t) = \lambda t \quad \lambda \in \mathbb{R}.
\end{equation*}
\invisible<2>{
  \vspace*{-0.2 cm}
  \item
    \textbf{Recherche d'une solution particulière} par la méthode de la variation de la constante.
    On cherche donc une solution particulière de $(E)$ notée $y_0$ sous la forme $y_0(t) = \lambda(t) t$ où $\lambda$ est une fonction dérivable sur $\mathbb{R}_{+}^{*}$.
    Alors, $y_0$ vérifie $(E)$ ssi,
    \begin{equation*}
t y_0^{\prime}(t) - y_0(t) = t^2 e^t \iff t (\lambda^{\prime}(t) t + \lambda(t)) - \lambda(t) t = t^2 e^t \iff \lambda^{\prime}(t) = e^t \iff \lambda(t) = e^t
    \end{equation*}
    \invisible<3>{
\item
\textbf{Conclusion :} l'ensemble des solutions de $(E)$ sont données par
    \begin{equation*}
y(t) = \lambda t + t e^t \quad \lambda \in \mathbb{R}.
    \end{equation*}
    \invisible<4>{
      }}}}
    \end{enumerate}
  }
  \end{frame}
%%%%%
\begin{frame}
  \frametitle{Résolution avec une condition initiale}
  Soient $a, b, c \in \mathcal{C}^{0}(I, \mathbb{K})$ et considérons l'EDO ${\bf a(t) y^{\prime}(t) + b(t) y(t) = c(t)}$
\invisible<1>{
  \begin{proposition}
    Si la fonction $a$ ne s'annule par sur $I$ et si $(t_0, y_0) \in I \times \mathbb{K}$, $ \exists ! f$ solution de $(E)$ tq $f(t_0) = y_0$.
  \end{proposition}
  \invisible<2>{
    \textbf{Démonstration :}
    \invisible<3>{
  Supposons qu'il existe $f_1$ et $f_2$ solutions de $(E)$ tq $f_1(t_0) = f_2(t_0) =  y_0$.
Alors,
\begin{equation*}
  \begin{split}
    a(t) f_1^{\prime}(t) + b(t) f_1(t) &= c(t)
    \\
    a(t) f_2^{\prime}(t) + b(t) f_2(t) &= c(t)
  \end{split}
  \Rightarrow a(t) (f_1 - f_2)^{\prime}(t) + b(t) (f_1 - f_2)(t) = 0.
\end{equation*}
\invisible<4>{
La solution de cette dernière EDO est donnée par $(f_1- f_2)(t) = \lambda e^{\dps -A(t)}$ où $A$ est une primitive de $\dps t \mapsto b(t) / a(t)$.
\invisible<5>{
Finalement $f_1(t) = \lambda e^{\dps -A(t)} + f_2(t)$. 
Or $f_1(t_0) = f_2(t_0) = y_0$ ce qui prouve que $\lambda = 0$. On en déduit que $f_1 = f_2$.
\invisible<6>{
  }}}}}}
  \end{frame}
%%%%%
\begin{frame}

   \begin{proposition}
    \begin{enumerate}
    \item
      Soit $\alpha \in \mathbb{C}$.
      La fonction $t \mapsto e^{\alpha t}$ est la seule fonction dérivable sur $\mathbb{R}$ vérifiant l'edo $y^{\prime} = \alpha y$ et la condition initiale $y(0) = 1$.
      \vspace*{0.2 cm}
    \item
      Les fonctions dérivables de $\mathbb{R}$ dans $\mathbb{C}$ vérifiant l'équation fonctionnelle
      \begin{equation}
\forall (t, u) \in \mathbb{R}^2, \ f(t+u) = f(t) f(u)
      \end{equation}
      sont la fonction nulle et les fonctions $t \mapsto e^{\alpha t}$ avec $\alpha \in \mathbb{C}$.
      \end{enumerate}
   \end{proposition}
   \invisible<1>{
     \textbf{Démonstration :}
     \invisible<2>{
    \begin{enumerate}
    \item
      Le problème de Cauchy
      \begin{equation*}
          y^{\prime}(t) = \alpha y(t), \quad y(0) = 1
      \end{equation*}
      admet une unique solution.
      Puisque la fonction $t \mapsto e^{\alpha t}$ est solution il s'agit de la seule fonction vérifiant ce problème.
      \end{enumerate}
    \invisible<3>{
      }}}
  \end{frame}
%%%%%
\begin{frame}
  \begin{enumerate}
\setcounter{enumi}{1}
    \item
      Soit $f$ une fonction dérivable vérifiant l'équation fonctionnelle
      \begin{equation*}
\forall (t, u) \in \mathbb{R}^2, \ f(t + u) = f(t) f(u).
      \end{equation*}
      \invisible<1>{
      Pour $t = u = 0$ on trouve
      \begin{equation*}
f(0) = \left[f(0)\right]^2 \Rightarrow f(0) = 1.
      \end{equation*}
            \invisible<2>{
      Ensuite, on dérive l'équation fonctionnelle par rapport à la variable $u$ et on trouve
      \begin{equation*}
f^{\prime}(t + u) = f(t) f^{\prime}(u).
      \end{equation*}
            \invisible<3>{
      Pour $u = 0$ on obtient comme relation
      \begin{equation*}
f^{\prime}(t) = f(t) f^{\prime}(0).
      \end{equation*}
            \invisible<4>{
      Finalement, la fonction $f$ est solution du problème de Cauchy
      \begin{equation*}
y^{\prime} = \alpha y \quad \text{où} \quad \alpha = f^{\prime}(0) \quad \text{et} \quad y(0) = 1.
      \end{equation*}
            \invisible<5>{
              Or la solution de ce problème est donnée par $t \mapsto e^{\alpha t}$ ce qui donne le résultat.
              \invisible<6>{
                }}}}}}
      \end{enumerate}
  \end{frame}
%%%%%
\begin{frame}
\frametitle{\'{E}quations du second ordre à coefficients constants}
\begin{enumerate}
\item
  \textbf{Résolution de l'équation homogène}
  \vspace*{0.2 cm}
  \\
  Soient $a$, $b$, et $c$ avec $a \neq 0$ trois nombres complexes ou réels.
  On considère l'équation différentielle :
  \begin{equation*}
a y^{\prime \prime}(t) + b y^{\prime}(t) + c y(t) = 0  \qquad \qquad (E_0).
  \end{equation*}
  On s'intéresse à la résolution de l'équation différentielle $(E_0)$.
  \begin{definition}
    On appelle équation caractéristique associée à l'équation différentielle $(E_0)$ l'équation :
    \begin{equation*}
ar^2 + br + c = 0.
      \end{equation*}
  \end{definition}

  \end{enumerate}
\end{frame}
%%%%%%
\begin{frame}
\begin{proposition}
Soit $r \in \mathbb{C}$. La fonction $\varphi_r : t \mapsto e^{rt}$ est solution de $(E_0)$ si, et seulement si, $a r^2 + br + c = 0$.
\end{proposition}
      \invisible<1>{
        \textbf{Démonstration :}
      \invisible<2>{
        Par double implication.
\begin{itemize}
\item[$\bullet$]
  \invisible<3>{
    Supposons que la fonction $\varphi_r$ est solution de $(E_0)$. En injectant l'expression de $\varphi_r$ dans $(E_0)$ on trouve
        \begin{equation*}
\left(a r^2+ b r + c\right)  e^{rt} = 0
        \end{equation*}
        Nécessairement $a r^2+ b r + c = 0$.
        \\
        \vspace*{0.25 cm}
              \invisible<4>{
      \item[$\bullet$]
        Réciproquement, si $a r^2+ b r + c = 0$ alors on a $e^{rt} \left( a r^2+ b r + c = 0 \right)$ et donc $a \varphi_r^{\prime \prime}(t) + b \varphi_{r}^{\prime}(t) + c \varphi_r(t) = 0$.
\end{itemize}
\invisible<5>{
  }}}}}
  \end{frame}
%%%%%
\begin{frame}
  \frametitle{Calcul des solutions}
\begin{proposition}[Cas complexe]
    Soit $(a, b, c) \in \mathbb{C}^3$.
    \begin{itemize}
      \item[$\bullet$]
        Si l'équation caractéristique $a r^2 + br + c$ a deux racines distinctes $r_1$ et $r_2$, les solutions de $(E_0)$ sont les fonctions de la forme :
        \begin{equation*}
t \mapsto \lambda_1 e^{r_1 t} + \lambda_2 e^{r_2 t} \quad \text{avec} \quad (\lambda_1, \lambda_2) \in \mathbb{C}^2
        \end{equation*}
      \item[$\bullet$]
        Si l'équation caractéristique $a r^2 + br + c$ a une racine double $r_0$, les solutions de $(E_0)$ sont les fonctions de la forme :
        \begin{equation*}
t \mapsto e^{r_0 t} \left( \lambda_1 + \lambda_2 t \right) \quad \text{avec} \quad (\lambda_1,\lambda_2) \in \mathbb{C}^2.
        \end{equation*}
        \end{itemize}
\end{proposition}
  \end{frame}
%%%%%%%
\begin{frame}
\textbf{Démonstration :}
    \begin{itemize}
    \item[$\bullet$]
            \invisible<1>{
      Supposons que l'équation caractéristique $ar^2 + br + c$ a deux racines distinctes $r_1$ et $r_2$.
      Alors, les fonctions $\varphi_{r_1}$ et $\varphi_{r_2}$ sont solutions de $(E_0)$.
      \textcolor{red}{L'espace des solutions de $(E_0)$ est un espace vectoriel !} ce qui prouve le résultat.
      \vspace*{0.2 cm}
      \\
    \item[$\bullet$]
            \invisible<2>{
      Supposons que l'équation caractéristique $a r^2 + br + c$ a une racine double notée $r_0$.
      Alors, $r_0 = -\frac{b}{2a}$.
      \\
            \invisible<3>{
              Soit $f$ la fonction définie sur $\mathbb{R}$ par $f(t) = \lambda_1 + \lambda_2 t$.
                    \invisible<4>{
      La fonction $t \mapsto f(t) e^{r_0t}$ est solution de l'équation $(E_0)$ si, et seulement si, elle est deux fois dérivables et
      \begin{equation*}
        \begin{split}
         & a (f(t)e^{r_0t})^{\prime \prime} + b (f(t)e^{r_0t})^{\prime} + c (f(t) e^{r_0t}) = 0
          \\
        %  \iff & a \left(f^{\prime}(t) e^{rt} + r f(t) e^{rt}  \right)^{\prime} + b \left(f^{\prime}(t) e^{rt} + r f(t) e^{rt} \right) + c f(t) e^{rt} = 0
          %% \\
          %% \iff
          %% a \left(f^{\prime \prime}(t) e^{rt} + 2r f^{\prime}(t)e^{rt}  + r^2f(t)e^{rt}  \right) + b \left(f^{\prime}(t) e^{rt} + r f(t) e^{rt} \right) + c f(t) e^{rt} = 0
          \iff
         & \left(a r_0^2 + br_0 + c\right) f(t) e^{r_0t} + (2ar_0 + b)f^{\prime}(t) e^{r_0t} + a f^{\prime \prime}(t) e^{r_0t} = 0
          \\
         % \iff
         %& (2ar_0 + b)f^{\prime}(t) e^{r_0t} + a f^{\prime \prime}(t) e^{r_0 t} = 0
          %\\
          \iff
          & a f^{\prime \prime}(t) e^{r_0 t} = 0
          \\
          \iff
          & f^{\prime \prime}(t) = 0
          \iff
           f(t) = \lambda_1 + \lambda_2 t \quad \text{où} \quad (\lambda_1, \lambda_2) \in \mathbb{R}^2.
          \end{split}
        \end{equation*}
      \invisible<5>{
        }}}}}
      \end{itemize}
  \end{frame}
%%%%%
\begin{frame}
  \vspace*{-0.5 cm}
 \begin{proposition}[Cas réel]
    Soient $a$, $b$, et $c$ trois réels.
    \begin{itemize}
      \invisible<1>{
    \item[$\bullet$]
        Si l'équation caractéristique $a r^2 + br + c$ a deux racines réelles distinctes $r_1$ et $r_2$, les solutions de $(E_0)$ sont les fonctions de la forme :
        \begin{equation*}
t \mapsto \lambda_1 e^{r_1 t} + \lambda_2 e^{r_2 t} \quad \text{avec} \quad (\lambda_1, \lambda_2) \in \mathbb{R}^2
        \end{equation*}
        \vspace*{-0.5 cm}
              \invisible<2>{
      \item[$\bullet$]
        Si l'équation caractéristique $a r^2 + br + c$ a une racine double $r_0$, les solutions de $(E_0)$ sont les fonctions de la forme :
        \begin{equation*}
t \mapsto e^{r_0 t} \left( \lambda_1 + \lambda_2 t \right) \quad \text{avec} \quad (\lambda_1,\lambda_2) \in \mathbb{R}^2.
        \end{equation*}
        \vspace*{-0.5 cm}
              \invisible<3>{
      \item[$\bullet$]
        Si l'équation caractéristique $a r^2 + br + c$ a deux racines complexes conjuguées distinctes $r_1 = \alpha + i \beta$ et $r_2 = \alpha - i \beta$ les solutions de $(E_0)$ sont les fonctions :
        \begin{equation*}
t \mapsto \gamma_1 e^{\alpha t} \cos(\beta t) + \gamma_2 e^{\alpha t} \sin(\beta t) \quad \text{avec} \quad (\gamma_1, \gamma_2) \in \mathbb{R}^2
        \end{equation*}
        \invisible<4>{
          }}}}
    \end{itemize}
 \end{proposition}
\end{frame}
%%%%
\begin{frame}
  \textbf{Démonstration :}
  \begin{itemize}
          \invisible<1>{
    \item[$\bullet$]
      Les deux premiers points se démontrent à l'instar du cas complexe.
      \invisible<2>{
    \item[$\bullet$]
        Supposons que l'équation caractéristique $a r^2 + br + c$ a deux racines complexes conjuguées distinctes $r_1 = \alpha + i \beta$ et $r_2 = \alpha - i \beta$.
              \invisible<3>{
        Les fonctions $\varphi_{r_1}$ et $\varphi_{r_2}$ sont solutions de l'équation $(E_0)$.
              \invisible<4>{
    Comme $(E_0)$ a une structure d'espace vectoriel $\exists (\lambda_1,\lambda_2) \in \mathbb{R}^2$ tel que $\lambda_1 \varphi_{r_1} + \lambda_2 \varphi_{r_2}$ est solution de $(E_0)$.
    Finalement 
    \begin{equation*}
\lambda_1 e^{r_1 t} + \lambda_2 e^{r_2 t} = 0
    \end{equation*}
    et donc
    \begin{equation*}
      \begin{split}
      \lambda_1 e^{(\alpha + i \beta)t} + \lambda_2 e^{(\alpha - i \beta)t}
      & =  e^{\alpha t} \left( (\lambda_1 + \lambda_2) \cos(\beta t) +  i (\lambda_1 - \lambda_2) \sin(\beta t)  \right)
      \\
      & = e^{\alpha t} \left( \gamma_1 \cos(\beta t) +  i \gamma_2 \sin(\beta t)  \right) = 0.
      \end{split}
    \end{equation*}
    \invisible<5>{
      }}}}}
    \end{itemize}
  \end{frame}
%%%
\begin{frame}
  \frametitle{Exercice}
        \invisible<1>{
    Résoudre l'équation différentielle suivante :
    \begin{equation*}
y^{\prime \prime}(t) - 2 y^{\prime}(t) + y(t) = 0
    \end{equation*}
          \invisible<2>{
    \corrige{
  Il s'agit d'une équation différentielle linéaire homogène du second ordre à coefficients constants.
    Son équation caractéristique associée est
    \begin{equation*}
r^2 -2r + 1 = 0.
    \end{equation*}
    Le discriminant $\Delta$ de cette équation est $\Delta = 0$. Cette équation admet donc une racine double $r = 1$.
    Finalement les solutions de l'équation différentielle sont les fonctions de la forme $y(t)= e^{t} \left(A + Bt \right)$ où $(A, B) \in \mathbb{R}^2$.
    }
    \invisible<3>{
      }}}
  \end{frame}
%%%
\begin{frame}
  \begin{enumerate}
    \setcounter{enumi}{1}
  \item \textbf{Résolution de l'équation non homogène}
    \vspace*{0.4 cm}
    \\
     Considérons l'équation différentielle du second ordre
  \begin{equation*}
a y^{\prime \prime}(t) + b y^{\prime}(t) + c y(t)= d(t) \qquad (E).
  \end{equation*}
 \textcolor{cadmiumgreen}{Solutions de $(E)$ : solution de l'équation homogène $(E_0)$ $+$ solution particulière.}
 \vspace*{0.3 cm}
 \\
 \textcolor{red}{\textbf{Comment trouver une solution particulière ?}}
\vspace*{0.3 cm}
\\
$\rightarrow$ Le second membre donne des informations : Polynôme, exp etc...
    \end{enumerate}
\end{frame}
%%%%
\begin{frame}
\frametitle{Cas où la fonction $d$ est un polynôme}
\textbf{On cherche une solution polynomiale !}
 \vspace*{0.2 cm}
 \\
Remarque : si $y$ est un polynôme de degré $p$ alors $a y^{\prime \prime} + b y^{\prime} + c y$ est un polynôme
\begin{itemize}
\item[$\bullet$]
  de degré $p$ si $c \neq 0$
\item[$\bullet$] de degré $p-1$ si $c=0$ et $b\neq 0$
  \item[$\bullet$] de degré $p-2$ si $c=b=0$.
  \end{itemize}
\begin{proposition}
  Soit $P$ un polynôme de degré $n$. On considère l'EDO ${\bf ay^{\prime \prime}(t) + b y^{\prime}(t) + c y(t) = P(t) \qquad (E)}$.
  \vspace*{0.2 cm}
  \\
  L'équation $(E)$ possède comme solution particulière une fonction polynomiale de degré :
  \begin{itemize}
  \item[$\bullet$]
$n$ si $c \neq 0$
  \item[$\bullet$]
    $n+1$ si $c=0$ et $b \neq 0$
  \item[$\bullet$]
    $n+2$ si $b = c = 0$.
    \end{itemize}
\end{proposition}

  \end{frame}
%%%
\begin{frame}
  \textbf{Démonstration :}
\begin{itemize}
  \item[$\bullet$]
   Soit $c \neq 0$ et soit $P$ un polynôme de degré $n$ s'écrivant sous la forme
$\dps P(t) = \sum_{k=0}^{n} \alpha_k t^k$.
    Soit $y$ une fonction de la forme $\dps y(t) = \sum_{k=0}^{n} \beta_k t^k$.
    La fonction $y$ est dérivable et on a
    \begin{equation*}
y^{\prime}(t) = \sum_{k=1}^{n} k \beta_k  t^{k - 1} \quad \text{et}\quad y^{\prime \prime}(t) = \sum_{k=2}^{n} k (k - 1) \beta_k  t^{k - 2}.
    \end{equation*}
    Aussi, \textcolor{cadmiumgreen}{$y$ est solution de $(E)$ si, et seulement si,}
    \begin{equation*}
      \begin{split}
        &        a \sum_{k=2}^{n} k (k - 1) \beta_k  t^{k - 2} + b \sum_{k=1}^{n} k \beta_k  t^{k - 1} + c \sum_{k=0}^{n} \beta_k t^k = \sum_{k=0}^{n} \alpha_k t^k
        \\
&        \iff
c \beta_0 + c \beta_1 t + b \beta_1 + \hspace*{-0.1 cm} \sum_{k=2}^{n} \left( k(k-1) \beta_k t^{k-2} + b  k \beta_k t^{k-1} + c  \beta_k t^k \right) = \alpha_0 + \alpha_1 t + \hspace*{-0.1 cm} \sum_{k=2}^{n} \alpha_k t^{k}        
      \end{split}
    \end{equation*}
    \end{itemize}
\end{frame}
\begin{frame}
  \begin{itemize}
  \item
    par identification on obtient le système matriciel
    %% \begin{equation}
    %%   \label{eq:systeme:mat}
    %%   \begin{split}
    %%   c \beta_0 + b \beta_1 + 2 \beta_2 &= \alpha_0
    %%   \\
    %%   c \beta_1 + 2 b \beta_2 + 6 \beta_3 &= \alpha_1
    %%   %\\
    %%   %c \beta_2 + 3b \beta_3 + 12 \beta_4 &= \alpha_2
    %%   \\
    %%    & \vdots
    %%  % \\
    %%   %n(n-1)\beta_n + b(n-1) \beta_{n-1} + c \beta_{n-2} &= \alpha_{n-2} t^{n-2}
    %%   %\\
    %%  % b n \beta_n + c \beta_{n-1} &= \alpha_{n-1}
    %%   \\
    %%   c \beta_n &= \alpha_n
    %%   \end{split}
    %% \end{equation}
%    Puis sous forme matricielle 
    \begin{equation*}
%      \label{eq:systeme:matriciel:edo}
\underbrace{
      \begin{pmatrix}
        c & b & 2 & \cdots & 0              & 0 & 0 & \cdots & 0
        \\
        0 & c & 2b & 6 & 0                  & 0 & 0 & \cdots & 0
        \\
        0 & 0 & c & 3b & 12                 & 0 & 0 & \cdots & 0
        \\
        \vdots & \vdots & \vdots & \vdots & \vdots & \vdots & \vdots & \vdots & \vdots
        \\
        \vdots & \vdots & \vdots & \vdots & \vdots & \vdots & \vdots & \vdots & \vdots
                \\
                \vdots & \vdots & \vdots & \vdots & \vdots & \vdots & \vdots & \vdots & \vdots
                \\
                0 & 0 & 0 & 0 & 0 & 0  &   c & b (n-1) & n(n-1)
                \\
                0 & 0 & 0 & 0 & 0 & 0  & 0 & c & b n
                \\
  0 & 0 & 0 & 0 & 0 & 0  & 0  & 0 & c
      \end{pmatrix}
      }_{\textcolor{red}{\mathbb{A}}}
\underbrace{
\begin{pmatrix}
        \beta_0
        \\
        \beta_1
        \\
        \beta_2
        \\
        \vdots
        \\
        \vdots
        \\
        \vdots
        \\
        \beta_{n-2}
        \\
        \beta_{n-1}
        \\
        \beta_n
\end{pmatrix}
}_{{\bm X}}
=
\underbrace{
      \begin{pmatrix}
        \alpha_0
        \\
        \alpha_1
        \\
        \alpha_2
        \\
        \vdots
        \\
        \vdots
        \\
        \vdots
        \\
        \alpha_{n-2}
        \\
        \alpha_{n-1}
        \\
        \alpha_n
      \end{pmatrix}
}_{\bm B}
    \end{equation*}
Ainsi, $y$ est une solution de $(E)$ si, et seulement si $\det(\mathbb{A}) \neq 0$.
On en conclue bien que $y$ est solution de $(E)$.
\end{itemize}
  \end{frame}
  %%%%
\begin{frame}
\begin{itemize}
\item[$\bullet$]
    Supposons que $c = 0$ et $b \neq 0$. L'équation $(E)$ devient
    \begin{equation*}
a y^{\prime \prime}(t) + b y^{\prime}(t) = P(t).
      \end{equation*}
    En posant $z(t) = y^{\prime}(t)$ on obtient une équation différentielle du 1er ordre d'inconnue $z(t)$ :
    \begin{equation*}
a z^{\prime}(t) + b z (t) = P(t)
    \end{equation*}
    $P$ étant un polynôme de degré $n$ il vient que $a z^{\prime}(t) + b z(t)$ est un polynôme de degré $n$ et donc que $z$ est un polynôme de degré $n$.
    Finalement, $y$ est bien un polynôme de degré $n+1$.
  \item[$\bullet$]
    Supposons que $b = c = 0$.
    L'équation $(E)$ s'écrit
    \begin{equation*}
a y^{\prime \prime}(t) = P(t).
    \end{equation*}
    Comme $a$ est constant et que $P$ est de degré $n$, en intégrant l'équation précédente il vient que $y$ est un polynôme de degré $n + 2$.
    \end{itemize}
    \end{frame}
%%%%%%%%%%
\begin{frame}
  \frametitle{Cas où $d$ est un produit de polynôme exponentiel}
  On suppose que $d$ est de la forme $d(t) = P(t) e^{mt}$ où $m \in \mathbb{C}$.
  \vspace*{0.2 cm}
  \\
      \invisible<1>{
  \textcolor{cadmiumgreen}{\textbf{Etude préliminaire:}}
\vspace*{0.2 cm}
\\
\begin{enumerate}
      \invisible<2>{
\item
Changement de variable : on pose $y(t) = e^{mt} z(t)$.
L'équation $(E)$ se réecrit alors
\begin{equation*}
  \begin{split}
    a \left( \left( m^2 e^{mt} z(t) + m e^{mt} z^{\prime}(t)  \right)             +m e^{mt} z^{\prime}(t) + e^{mt} z^{\prime \prime}(t) \right) &
    \\
    + b \left(m e^{mt} z(t) + e^{mt} z^{\prime}(t)  \right) + c e^{mt} z(t) = P(t) e^{mt}
    \end{split}
\end{equation*}
En simplifiant par $e^{mt}$ on obtient
%% \begin{equation*}
%% a \left( \left( m^2 z(t) + m z^{\prime}(t)  \right) + m z^{\prime}(t) +  z^{\prime \prime}(t) \right) + b \left(m  z(t) +  z^{\prime}(t)  \right) + c z(t) = P(t) 
%% \end{equation*}
soit
\begin{equation*}
a z^{\prime \prime}(t) + (2am + b) z^{\prime}(t) + (a m^2 + bm + c) z(t) = P(t) 
\end{equation*}
      \invisible<3>{
\item
On obtient une EDO du 2nd ordre de variable $z$ avec un second membre polynomial. \textcolor{red}{on trouve la solution $y$ !}
\invisible<4>{
   }}}}
\end{enumerate}
\end{frame}
%%%%%
\begin{frame}
\begin{proposition}
  Soit $m \in \mathbb{C}$ et $P$ un polynôme de degré $n$. On peut trouver une solution particulière de l'équation :
  \begin{equation}
a y^{\prime \prime}(t) + b y^{\prime}(t) + cy(t) = e^{mt} P(t)
  \end{equation}
  de la forme $y(t) = e^{mt} Q(t)$ où $Q$ est un polynôme :
  \vspace*{0.2 cm}
\\
  \begin{itemize}
  \item[$\bullet$] de degré $n$ si $m$ n'est pas racine de l'équation $a r^2 + br + c = 0$,
  \vspace*{0.2 cm}
\\  
\item[$\bullet$] de degré $n+1$ si $m$ est racine simple de l'équation $a r^2 + br + c = 0$
  \vspace*{0.2 cm}
\\
    \item[$\bullet$] de degré $n+2$ si $m$ est racine doucle de l'équation $ar^2 + br + c$.
    \end{itemize}

\end{proposition}

  \end{frame}
%%%
\begin{frame}
  \frametitle{Exercice}
  Résoudre l'équation différentielle suivante ${\bf y^{\prime \prime} - 4 y^{\prime} + 3 y = (2x + 1)e^{-x}}$ (E)

  \corrige{
    \begin{enumerate}
      \invisible<1>{
    \item
      \textbf{Résolution de l'équation homogène associée $(E_0)$ :}
      \\
      L'équation caractéristique est $r^2 - 4 r + 3$. Elle admet deux solutions $r_1 = 1$ et $r_2 = 3$.
      Les solutions de $(E_0)$ sont de la forme $t \mapsto \lambda_1 e^{t} + \lambda_2 e^{3t}$.
            \invisible<2>{
    \item \textbf{Calcul de la solution particulière $y_P$}
      \\
      Le 2nd membre est un produit polynôme-exponentiel.
      Comme $1$ n'est pas racine de l'eqn caractéristique on cherche $y_P$ de degré $1$ tq $y_P(x) = e^{-x}(a x + b)$.
      Or, $y_P$ est solution de $(E)$ ssi
      \begin{equation*}
        y_P^{\prime \prime}(x) - 4 y_P^{\prime}(x) + 3 y_P(x) = (2x + 1)e^{-x} \iff
       a = \frac{1}{4} \quad \text{et} \quad b = \frac{5}{16} 
      \end{equation*}
            \invisible<3>{
    \item
      \textbf{Conclusion :}
      $y(x) = \lambda_1 e^{x} + \lambda_2 e^{3x} + e^{-x}(\frac{1}{4}x + \frac{5}{16})$
      \invisible<4>{
        }}}}
      \end{enumerate}
  }
  \end{frame}
