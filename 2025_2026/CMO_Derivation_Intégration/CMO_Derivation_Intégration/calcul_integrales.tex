\begin{frame}
    \begin{center}
        \Huge{Calcul d'intégrales}
    \end{center}
\end{frame}

%------------------------------------------------

\begin{frame}{Primitives et intégrale d'une fonction continue}
\begin{definition}
    Soit $f \in \mathcal{C}^{0}(I, \mathbb{R})$. 
    On appelle primitive de $f$ sur $I$ toute fonction de $I$ dans $\mathbb{R}$, dérivable sur $I$ et dont la dérivée est égale à $f$.
\end{definition}
\textbf{Exemples :}
\vspace*{0.2 cm}
\begin{itemize}
   \invisible<1>{
    \item $f : \mathbb{R} \rightarrow \mathbb{R}$ définie par $f(x) = x^2$. Alors $g : \mathbb{R} \rightarrow \mathbb{R}$ définie par $g(x) = \frac{1}{3}x^3$ est dérivable sur $\mathbb{R}$ et $g^{\prime}(x) = f(x)$.
    \vspace*{0.2 cm}
    \invisible<2>{
    \item
    $f : \mathbb{R} \rightarrow \mathbb{R}$ définie par $f(x) = e^x$. 
    Alors $g : \mathbb{R} \rightarrow \mathbb{R}$ définie par $g(x) = e^x$ est dérivable sur $\mathbb{R}$ et $g^{\prime}(x) = f(x)$.
    \vspace*{0.2 cm}
    \invisible<3>{
    \item $f : \mathbb{R}_{*}^{+} \rightarrow \mathbb{R}_{*}^{+}$ définie par $f(x) = \sqrt{x}$. 
    Alors $g : \mathbb{R}_{+}^{*} \rightarrow \mathbb{R}_{+}^{*}$ définie par $g(x) = \frac{1}{2 \sqrt{x}}$ est dérivable sur $\mathbb{R}_{+}^{*}$ et $g^{\prime}(x) = f(x)$.
    \invisible<4>{
    }}}}    
\end{itemize}
\end{frame}
%%%%
\begin{frame}
\begin{proposition}
Soit $f \in \mathcal{C}^0(I)$. Si $F$ est une primitive de $f$, alors l'ensemble des primitives de $f$ sur $I$ sont les fonctions $F + \lambda$ avec $\lambda \in \mathbb{R}$. 
\end{proposition}
\invisible<1>{
\textbf{Démonstration :}
    Soit $G$ la fonction définie sur $I$ par 
\begin{equation*}
        G(x) = F(x) + \lambda \quad \forall x \in I.
\end{equation*}
    Alors, la fonction $G$ est dérivable sur $I$ et
$ G^{\prime}(x) = F^{\prime}(x) = f(x)$.
\invisible<2>{
\begin{thm}[Fondamental]
Soient $f$ une fonction continue de $I$ dans $\mathbb{R}$ et $a$ un point de $I$. 
La fonction $F_a$ définie par :
\begin{equation*}
F_a(x) = \int_{a}^{x} f(t)\,\mathrm{dt}   
\end{equation*}
est une primitive de $f$ sur $I$. 
C'est l'unique primitive de $f$ qui s'annule en $a$.
\end{thm}
\invisible<3>{
}}}
\end{frame}
%%%%%%%%%%
\begin{frame}
\textbf{Démonstration :}
    On prouve que $F_a$ est dérivable sur $I$ et  $F_a^{\prime} = f$ sur $I$.
    Soit $x_0 \in I$.
    \begin{equation*}
       \left  \lvert \frac{F_a(x) - F_a(x_0)}{x - x_0} - f(x_0) \right  \rvert = \left  \lvert \frac{1}{x - x_0} \left( \int_{a}^{x} f(t)\,\mathrm{dt} - \int_{a}^{x_0} f(t)\,\mathrm{dt} \right) - f(x_0) \right  \rvert
                 \leq \sup_{t \in [x_0, x]} \left \lvert f(t) - f(x_0) \right \rvert.
    \end{equation*}
Soit $\varepsilon > 0$. 
Comme $f$ est continue en $x_0$  
\begin{equation*}
 \exists \eta > 0 \ \forall t \in  ]x_0 - \eta, x_0 + \eta[ \cap [x_0,x],  \ \left \lvert f(t) - f(x_0) \right \rvert < \varepsilon \ \Rightarrow \sup_{t \in [x_0, x]} \myabs{f(t) - f(x_0)} < \varepsilon.
\end{equation*}
Alors,
\begin{equation*}
    \myabs{\frac{F_a(x) - F_a(x_0)}{x - x_0} - f(x_0)} 
\leq 
\varepsilon 
\end{equation*}
Donc
$F_a$ est dérivable en $x_0 \in I$ et $F_a^{\prime}(x_0) = f(x_0)$.
\end{frame}
%%%%%%%%
\begin{frame}
\begin{thm}
    Soit $f \in \mathcal{C}^{0}(I,\mathbb{R})$ et $a$ et $b$ deux points de $I$. 
    Si $F$ est une primitive de $f$ sur $I$, on a :
\begin{equation*}
    \int_{a}^{b} f(x)\,\mathrm{dx} = F(b) - F(a).
\end{equation*}    
\end{thm}
\vspace*{0.2 cm}
 \textbf{Exercices :}
 \vspace*{0.2 cm}
    Calculer les intégrales suivantes :
    \begin{enumerate}
        \item
        \begin{equation*}
        \dps \int_{a}^{b} e^{2x} \,\mathrm{dx} =  \textcolor{midnightblue}{\left[ \frac{1}{2} e^{2x}\right]_{a}^{b} = \frac{1}{2} \left( e^{2b} - e^{2a} \right)}    
        \end{equation*}
        \vspace*{0.2 cm}
                \item
            \begin{equation*}
                    \dps \int_{0}^{\pi} \sin(x)\,\mathrm{dx}= \textcolor{midnightblue}{\left[ -\cos(x) \right]_{0}^{\pi} = 2}
            \end{equation*}
    \end{enumerate}
\end{frame}
%%%%%
%%%%%%%%
\begin{frame}{Méthodes de calcul de primitives}
\invisible<1>{
    \begin{thm}[Intégration par parties]
Soit $(a,b) \in \mathbb{R}^2$. 
Soient $u$ et $v$ deux fonctions de classe $\mathcal{C}^1$ sur $[a,b]$. Alors
\begin{equation*}
    \int_{a}^{b} u(t) v^{\prime}(t)\,\mathrm{dt} = [u(t) v(t)]_a^b - \int_{a}^b u^{\prime}(t) v(t)\,\mathrm{dt}.
\end{equation*}
\end{thm}
\invisible<2>{
\textbf{Exercice :}
    Calculer l'intégrale suivante : $\dps \int_{1}^{2} \ln(x)\,\mathrm{dx}$.

\invisible<3>{
\corrige{$u : x \mapsto \ln(x)$ et $v: x \mapsto x$.
Alors $u$ et $v$ sont $\mathcal{C}^{1}([1,2])$. 
Par la formule d'IPP 
\begin{equation*}
    \int_{1}^{2} u(x) v^{\prime}(x)\,\mathrm{dx} = \left[u(x) v(x) \right]_{1}^{2} - \int_{1}^{2} u^{\prime}(x) v(x)\,\mathrm{dx} = \left[x \ln(x)\right]_{1}^{2} - \int_{1}^{2} 1 \,\mathrm{dx} = 2 \ln(2) - 1.
    \end{equation*}
}
\invisible<4>{
}}}}
\end{frame}
%%%%%
\begin{frame}
\vspace*{-0.3 cm}
\begin{thm}[Changement de variables]
    Soient $I$ et $J$ deux intervalles de $\mathbb{R}$ et $f \in \mathcal{C}^{0}(I,\mathbb{R})$ et $\varphi \in \mathcal{C}^{1}(J,I)$. 
Si $\alpha$ et $\beta \in J$ on a : 
\begin{equation*}
\int_{\varphi(\alpha)}^{\varphi(\beta)} f(t)\,\mathrm{dt} = \int_{\alpha}^{\beta} f(\varphi(u)) \varphi^{\prime}(u)\,\mathrm{du}.   
\end{equation*}
\end{thm}
\invisible<1>{
\textbf{Démonstration :}
  \textcolor{byzantine}{$f \in \mathcal{C}^{0}(I,\mathbb{R})$ donc admet une primitive $F$ d'après le Théorème fondamental}.
  \invisible<2>{
  De plus, $\varphi(\alpha) \in I$ et $\varphi(\beta) \in I$. Donc
  \begin{equation*}
\int_{\varphi(\alpha)}^{\varphi(\beta)} f(t)\,\mathrm{dt} 
    = F(\varphi(\beta)) - F(\varphi(\alpha)) = (F \circ \varphi)(\beta) - (F \circ \varphi)(\alpha).
  \end{equation*}
  \invisible<3>{
  Or $F$ est dérivable sur $I$ et $\varphi$ est de classe $\mathcal{C}^1$ sur $I$. Ainsi, 
  \begin{equation*}
\int_{\varphi(\alpha)}^{\varphi(\beta)} f(t)\,\mathrm{dt}        = \int_{\alpha}^{\beta} (F \circ \varphi)^{\prime}(u)\,\mathrm{du} = 
    \int_{\alpha}^{\beta} F^{\prime}(\varphi(u)) \varphi^{\prime}(u)\,\mathrm{du} = \int_{\alpha}^{\beta} f(\varphi(u)) \varphi^{\prime}(u)\,\mathrm{du}
  \end{equation*}
  \invisible<4>{
  }}}}
\end{frame}
%%%%%
\begin{frame}{Remarques}
    \alert{\textbf{On a le choix :}}
    \vspace*{0.4 cm}
    \begin{enumerate}
        \item Trouver la fonction  $\varphi$ de classe $ \mathcal{C}^1$ sous-jacente au changement de variable.
        \textbf{Avantage :} On voit tous les détails lors de la tranformation via la fonction $\varphi$ et c'est plus rigoureux. \textbf{Inconvénients :} Parfois un peu long.
        \vspace*{0.6 cm}
        \item
        On pose $u$ en fonction de la variable primale $x$ (par exemple) et on calcul $du$ en fonction de $dx$ et on adapte les bornes. 
        \textbf{Avantage :} Plus rapide. \textbf{Inconvénients :} moins rigoureux.
    \end{enumerate}
\end{frame}
\begin{frame}{Exercices}
    Calculer l'intégrale suivante :
\begin{equation*}
    \dps A =  \int_{0}^{\frac{\pi}{2}} \sin^2(u) \cos(u) \,\mathrm{du}
\end{equation*}  
\invisible<1>{
    \corrige{
        Soit $\varphi 
 \in \mathcal{C}^{\infty}([0,\frac{\pi}{2}])$ définie par $\varphi(u) = \sin(u)$ et $f \in \mathcal{C}^{\infty}([0,\frac{\pi}{2}])$ définie par $f(u)=u^2$. 
        Par la formule du changement de variable, on obtient
       \begin{equation*}
           A = \int_{0}^{\frac{\pi}{2}} f(\varphi(u)) \varphi^{\prime}(u)\,\mathrm{du} = \int_{\varphi(0)}^{\varphi(\frac{\pi}{2})} t^2 \,\mathrm{dt} = \int_{0}^{1} t^2 \,\mathrm{dt} 
           = \left[\frac{1}{3} t^3\right]_0^1 = \frac{1}{3}. 
      \end{equation*}
      }
      \invisible<2>{
\textcolor{cadmiumgreen}{\textbf{Autre rédaction possible :}} on pose $v(u) = \sin(u)$. 
Alors $\mathrm{dv} = \cos(u)\,\mathrm{du}$. Pour $u=0$ $\rightarrow \ v = 0$ et pour $u = \frac{\pi}{2} \rightarrow v = 1$.
      Donc
      \begin{equation*}
          A = \int_{0}^{1} v^2\,\mathrm{dt} = \frac{1}{3}.
      \end{equation*}
      \invisible<3>{
      }}}
       \end{frame}
       %%%%%%%
       \begin{frame}
       \frametitle{Exercices}
        Calculer $\dps B = \int_{-1}^{2} \sqrt{4 - u^2} u \,\mathrm{du}$.
        
        \invisible<1>{
        \corrige{
        Soit $\varphi \in \mathcal{C}^{1}([-1,2])$ définie par $\varphi(u) = u^2$ et $f \in \mathcal{C}^{0}([-1, 2])$ définie par $f(v) = \sqrt{4 - v}$. 
 Alors, on a 
        \begin{equation*}
            B = \frac{1}{2} \int_{-1}^{2} f(\varphi(u)) \varphi^{\prime}(u)\,\mathrm{du}
            =
            \frac{1}{2} \int_{\varphi(-1)}^{\varphi(2)} f(t)\,\mathrm{dt} = \frac{1}{2} \int_{1}^{4} \sqrt{4 - t}\,\mathrm{dt} = -\frac{1}{2} \left[\frac{2}{3} (4 - t)^{\frac{3}{2}} \right]_{1}^{4} = \sqrt{3}.
        \end{equation*}
        }
        \invisible<2>{
    \textcolor{cadmiumgreen}{\textbf{Autre rédaction possible :}}
        Posons $t = u^2$ de sorte que $dt = 2 u du$. 
        La formule du changement de variable donne
        \begin{equation*}
            B = \frac{1}{2} \int_{1}^{4} \sqrt{4 - t}\,\mathrm{dt} = -\frac{1}{2} \left[\frac{2}{3} (4 - t)^{\frac{3}{2}} \right]_{1}^{4} = \sqrt{3}. 
        \end{equation*}
        \invisible<3>{
        }}}
        \end{frame}
        %%%%%%%%%%
        \begin{frame}{Exercices}
        Calculer $\dps C = \int_{-1}^{\frac{1}{2}} \sqrt{1 - t^2}\,\mathrm{dt}$. 

        \corrige{
\invisible<1>{
                Soient $f$ et $\varphi$ les fonctions définies sur $[-1,\frac{1}{2}]$ par 
            $f : t \mapsto \sqrt{1 - t^2}$ et $\varphi : u \mapsto \sin(u)$. 
                Alors $f \in \mathcal{C}^{0}([-1, \frac{1}{2}])$, et  $\varphi \in \mathcal{C}^1([-1, \frac{1}{2}])$.
                \vspace*{0.2 cm}
                \\
         \invisible<2>{
         \textcolor{byzantine}{\textbf{Formule du changement de variable}}
\begin{equation*}
\begin{split}
\int_{-1}^{\frac{1}{2}} \sqrt{1 - t^2}\,\mathrm{dt}
& = \int_{\varphi(-\frac{\pi}{2})}^{\varphi(\frac{\pi}{6})} f(t)\,\mathrm{dt} = \int_{-\frac{\pi}{2}}^{\frac{\pi}{6}} f(\varphi(u))\varphi^{\prime}(u)\,\mathrm{du} 
\\
& = \int_{-\frac{\pi}{2}}^{\frac{\pi}{6}} \sqrt{1 - \sin^2(u)} \cos(u)\,\mathrm{du} = \int_{-\frac{\pi}{2}}^{\frac{\pi}{6}} \myabs{\cos(u)}  \cos(u)\,\mathrm{du} 
\\
& = \int_{-\frac{\pi}{2}}^{\frac{\pi}{6}}  \cos^2(u)\,\mathrm{du}
\end{split}
\end{equation*}
\invisible<3>{
}}}
}
\end{frame}
%%%%%%%%%%
\begin{frame}
\textcolor{midnightblue}{
    Ainsi,
\begin{equation*}
\begin{split}
\int_{-1}^{\frac{1}{2}} \sqrt{1 - t^2}\,\mathrm{dt} 
&=  \int_{-\frac{\pi}{2}}^{\frac{\pi}{6}} \frac{1}{2} \left( \cos(2u) + 1 \right)\,\mathrm{du} \ \textcolor{byzantine}{\mathrm{(Formule \ de \ Moivre)}}
\\
& = 
\frac{1}{2} \left[\frac{1}{2} \sin(2u) + u \right]_{-\frac{\pi}{2}}^{\frac{\pi}{6}} 
\\
&= \frac{1}{2} \left( \frac{\sqrt{3}}{4} + \frac{2 \pi}{3}\right) = \frac{\sqrt{3}}{8} + \frac{\pi}{3}.
\end{split}
\end{equation*}
}
        \end{frame}
%%%%%%%%%%%%%%%%%%
%%%%%%%%%%%%%%%%%%%
\begin{frame}{Changement de variable affine}
\invisible<1>{
\textcolor{cadmiumgreen}{\textbf{fonction périodique}}
\invisible<2>{
\begin{proposition}
Soit $f$ une fonction définie sur un intervalle $[a,b]$ et périodique de période $T > 0$. Alors
\begin{equation*}
\int_{a}^{b} f(u)\,\mathrm{du} = \int_{a+T}^{b+T} f(v)\,\mathrm{dv}   
\end{equation*}    
\end{proposition}
\invisible<3>{
\textbf{Démonstration :}
     Si $f$ est $T$-périodique alors 
       $\forall x \in [a,b], \ f(x + T) = f(x)$.
       
    Posons $\varphi(v) = v + T$.
    Alors, $\varphi \in \mathcal{C}^{1}([a,b])$. D'après le \textcolor{byzantine}{\textbf{théorème du changement de variable}}
\begin{equation*}  \int_{\varphi(a)}^{\varphi(b)} f(v)\,\mathrm{dv} = \int_{a}^{b} f(\varphi(u))\varphi^{\prime}(u)\,\mathrm{du}
 = \int_{a}^{b} f(u+T)\,\mathrm{du} = \int_{a}^{b} f(u)\,\mathrm{du}.
    \end{equation*}
    \invisible<4>{
    }}}}
\end{frame}
\begin{frame}
    \textcolor{cadmiumgreen}{\textbf{fonction paire}}
    \vspace*{0.2 cm}
\invisible<1>{
\begin{proposition} 
Soit $f$ une fonction continue sur un intervalle $I$ contenant $0$ et soit $a \in I$.
 Si $f$ est paire alors
 \begin{equation*}
   \dps \int_{-a}^{a} f(u)\,\mathrm{du} = 2 \int_{0}^{a} f(u)\,\mathrm{du}  
 \end{equation*}
\end{proposition}
\invisible<2>{
\vspace*{0.2 cm}
\textbf{Démonstration :}
\invisible<3>{
\begin{equation*}
\begin{split}
    \dps \int_{-a}^{a} f(u)\,\mathrm{du} 
         &= \int_{-a}^{0} f(u)\,\mathrm{du} + \int_{0}^{a} f(u)\,\mathrm{du}
         \\
         & = -\int_{0}^{-a} f(u)\,\mathrm{du} + \int_{0}^{a} f(u)\,\mathrm{du}
\end{split}
\end{equation*}
\invisible<4>{
    Soit $\varphi \in \mathcal{C}^1([-a,0])$ définie par 
    $\varphi(v) = -v$.
    \invisible<5>{
    }}}}}
    \end{frame}
    %%%%%%%%%%%%%%%
    \begin{frame}
    Par la formule du \textcolor{byzantine}{changement de variable}
    \invisible<1>{
\begin{equation*}
\int_{\varphi(-a)}^{\varphi(0)} f(u)\,\mathrm{du}  = \int_{-a}^{0} f(\varphi(u)) \varphi^{\prime}(u)\,\mathrm{du}=-\int_{-a}^{0} f(-u)\,\mathrm{du}= -\int_{-a}^{0} f(u)\,\mathrm{du}= \int_{0}^{-a} f(u)\,\mathrm{du}.
\end{equation*}
\invisible<2>{
Or 
\begin{equation*}
\int_{\varphi(-a)}^{\varphi(0)} f(u)\,\mathrm{du}  = \int_{a}^{0} f(u)\,\mathrm{du}
= - \int_{0}^{a} f(u)\,\mathrm{du} = \int_{0}^{-a} f(u)\,\mathrm{du}
\end{equation*}
\invisible<3>{
Donc
\begin{equation*}
- \int_{0}^{-a} f(u)\,\mathrm{du} = \int_{0}^{a} f(u) \,\mathrm{du}. 
\end{equation*}
\invisible<4>{
Ainsi
\begin{equation*}
    \int_{-a}^{a} f(u)\,\mathrm{du} = 2 \int_{0}^{a} f(u)\,\mathrm{du}.
\end{equation*}
\invisible<5>{
}}}}}
\end{frame}
%%%%%%
\begin{frame}
\vspace*{-0.3 cm}
    \textcolor{cadmiumgreen}{\textbf{fonction impaire}}
    \invisible<1>{
\begin{proposition}
Soit $f$ une fonction continue sur un intervalle $I$ contenant $0$ et $a \in I$. 
Si $f$ est impaire alors
\begin{equation*}
\dps \int_{-a}^{a} f(u)\, \mathrm{du} = 0    
\end{equation*}
\end{proposition} 
\invisible<2>{
Le \textcolor{byzantine}{changement de variable} précédent donne
\begin{equation*}
\int_{\varphi(-a)}^{\varphi(0)} f(u)\,\mathrm{du} = \int_{a}^{0} f(u)\,\mathrm{du} = \int_{-a}^{0} -f(-u) \,\mathrm{du} = \int_{-a}^{0} f(u)\,\mathrm{du}.
\end{equation*}
Ainsi,
\vspace*{-0.1 cm}
\begin{equation*}
    -\int_{0}^{a} f(u)\,\mathrm{du} = \int_{a}^{0} f(u)\,\mathrm{du} = \int_{-a}^{0} f(u)\,\mathrm{du}.
\end{equation*}
\begin{equation*}
\textcolor{red}{\Rightarrow}    \int_{-a}^{a} f(u)\,\mathrm{du} = 0.
\end{equation*}
\invisible<3>{
}}}
\end{frame}
%%%%%%%%%%%%%%%%%%
\begin{frame}
    \textcolor{cadmiumgreen}{\textbf{Transformation affine sur l'élément de référence}}
    \invisible<1>{
    \begin{proposition}
        Soit $f$ une fonction continue sur un intervalle $I$ et $a$ et $b$ deux points de $I$.
Alors,
\begin{equation*}
\int_{a}^{b} f(t)\,\mathrm{dt}  = (b - a) \int_{0}^1 f(a + (b-a)v) \,\mathrm{dv}
\end{equation*}
    \end{proposition}
    \invisible<2>{
    \textbf{Démonstration :}
Soit $\varphi \in \mathcal{C}^1([0,1])$ définie par 
    \begin{equation*}
        \varphi(v) = a + (b-a)v.
    \end{equation*}
    D'après la formule du \textcolor{byzantine}{changement de variable}
    \begin{equation*}
\int_{\varphi(0)}^{\varphi(1)} f(t)\,\mathrm{dt} = \int_{0}^{1} f(\varphi(v)) \varphi^{\prime}(v)\,\mathrm{dv}
    = (b - a) \int_{0}^{1} f(a + (b - a)v) \,\mathrm{dv}.
    \end{equation*}
    \invisible<3>{
    }}}
\end{frame}
%%%%%%%%%%%%%
\begin{frame}{Primitives des fonctions polynômes-exponentielles}
\invisible<1>{
\begin{proposition}
    Soit $a \in \mathbb{C}^{*}$ et $P$ une fonction polynomiale.
    Alors la fonction $x\mapsto P(x) e^{ax}$ a une primitive de la forme $x \mapsto Q(x) e^{ax}$ où $Q$ est une fonction polynomiale de même degré que $P$.
\end{proposition}
\invisible<2>{
\textbf{Exercice :}
    Déterminer une primitive de la fonction $f$ définie sur $\mathbb{R}$ par
    \begin{equation*}
        f(x) = e^{x} (2x^3 + 3x^2 - x + 1)
    \end{equation*}
    \invisible<3>{
\corrige{
$f$ est le produit d'un polynôme de degré $3$ et d'une exponentielle. On cherche donc une primitive s'écrivant sous la forme 
\begin{equation*}
    F(x) = e^{x} Q(x) \quad \text{où} \ Q \in \mathbb{R}_3[X]  \ \text{i.e.} \ Q(x) = ax^3 + bx^2 + cx + d
\end{equation*}
On a 
\begin{equation*}
    F^{\prime}(x) = e^x (ax^3 + bx^2 + cx + d) + e^x (3 a x^2 + 2bx + c) = f(x).
\end{equation*}
}
\invisible<4>{
}}}}
\end{frame}
%%%%%%%%%%%%%%
\begin{frame}
\textcolor{midnightblue}{
Alors on a 
\begin{equation*}
\begin{split}
    F^{\prime}(x) &= x^3(a e^x) + x^2(b e^{x} + 3a e^x) + x(c e^x + 2b e^x) + (d+c)e^x
    \\
    & = 2e^x x^3 + 3x^2 e^x - x e^x + e^x
    \end{split}
\end{equation*}
\invisible<1>{
Par identification,
\begin{equation*}
    \begin{split}
        a & = 2
        \\
        3a + b & = 3
        \\
        c + 2b & = -1
        \\
        d + c &= 1.
    \end{split}
\end{equation*}
D'où,
\begin{equation*}
    a = 2, \quad b = -3, \quad c = 5, \quad d = -4.
\end{equation*}
\invisible<2>{
Finalement
\begin{equation*}
    \int f(x)\,\mathrm{dx} = e^x (2 x^3 - 3 x^2 + 5x -4) + k \quad k \in \mathbb{R}.
\end{equation*}
\invisible<3>{
}}}
}
\end{frame}
%%%%%%%%%%%%
%%%%%%%%%%%%%%%%
\begin{frame}{Primitives d'une fraction rationnelle}

\begin{proposition}
Soit $I$ un intervalle de $\mathbb{R}$ et soit $a \in \mathbb{R}$.
\begin{enumerate}
    \item 
    Si $a \not \in I$
     alors
     \begin{equation*}
     \int \frac{1}{x-a}\,\mathrm{dx} = \ln \myabs{x-a} + k \quad k \in \mathbb{R}
     \end{equation*}
   \item
   Si $a \in \mathbb{C}$ tel que $a = \alpha + i \beta$, $(\alpha, \beta) \in \mathbb{R} \times \mathbb{R}^{*}$, alors sur tout intervalle $I$ de $\mathbb{R}$ on a 
\begin{equation*}
 \int \frac{1}{x-a}\,\mathrm{dx}
= \frac{1}{2} \ln((x - \alpha)^2 + \beta^2) + i \arctan\left(\frac{x - \alpha}{\beta}\right) + k, \quad k \in \mathbb{R}.
\end{equation*}
\end{enumerate}    
\end{proposition}
\textbf{Démonstration :}
\begin{enumerate}
    \item 
        Trivial. Il suffit de dériver le membre de droite de l'équation.
\end{enumerate}
\end{frame}
%%%%%
\begin{frame}
\begin{enumerate}
\setcounter{enumi}{1}
    \item 
    Soit $a \in \mathbb{C}$ tel que $a = \alpha + i \beta$, $\alpha \in \mathbb{R}$, $\beta \in \mathbb{R}^{*}$.
    On a 
    \begin{equation*}
    \begin{split}
    \frac{1}{x - a} & =     \frac{x - \overline{a}}{(x - a)(x - \overline{a})} 
    = \frac{x - (\alpha - i \beta)}{(x - (\alpha + i \beta))(x - (\alpha - i \beta))} 
    = \frac{x - \alpha + i \beta}{ (x - \alpha)^2 + \beta^2 }
    \\
    & = \frac{x - \alpha}{ (x - \alpha)^2 + \beta^2 } + i \frac{1}{\beta}\frac{1}{ 1 + \frac{(x - \alpha)^2}{\beta^2} }
    \end{split}
    \end{equation*}
    En intégrant la dernière équation on obtient
   \begin{equation*}
       \int \frac{1}{x-a}\,\mathrm{dx}
= \frac{1}{2} \ln((x - \alpha)^2 + \beta^2) + i \arctan\left(\frac{x - \alpha}{\beta}\right) + k, \quad k \in \mathbb{R}
   \end{equation*} 
\end{enumerate}
    
\end{frame}

%%%%%%%%%%%%
\begin{frame}{Exercice}
Calculer une primitive de la fonction $f$ définie par
            $\dps f(x) = \frac{1}{(x^2 - 1)(x - 2)^2}$

        \corrige{
        \invisible<1>{
    \begin{equation*}
        \mathcal{D}_f = \left] -\infty, -1 \right[ \cup \left] -1, 1
 \right[ \cup \left]1, 2 \right[ \cup \left]2, + \infty \right[
    \end{equation*}
    \invisible<2>{
    $f$ admet $-1$ et $1$ comme pôles d'ordre $1$, et $2$ comme pôle d'ordre $2$.
    Par le théorème de la décomposition en éléments simples, 
\begin{equation*}
    f(x) = \frac{1}{2}\left( \frac{1}{x - 1}\right) - \frac{1}{18} \left(\frac{1}{x + 2}\right) - \frac{4}{9} \left( \frac{1}{x - 2} \right) + \frac{1}{3} \left(\frac{1}{(x - 2)^2}\right)
\end{equation*}
\invisible<3>{
Une primitive de la fonction $f$ est donc 
\begin{equation*}
    F(x) = \frac{1}{2} \ln{\myabs{x - 1}} - \frac{1}{18} \ln{\myabs{x + 2}} - \frac{4}{9} \ln{\myabs{x - 2}} - \frac{1}{3} \frac{1}{x - 2} + k, \quad k \in \mathbb{R}
\end{equation*}
\invisible<4>{
}}}}
        }
\end{frame}
%%%
\begin{frame}{Exercice}
Calculer une primitive de la fonction $f$ définie par 
            $\dps f(x) = \frac{5}{x^2 + x + 1 + i}$

\corrige{
\invisible<1>{
$-i$ et $-i + 1$ sont pôles d'ordre 1 de $f$. 
Par le théorème de la décomposition en éléments simples
\begin{equation*}
    f(x) = \frac{1 + 2i}{x + 1} - \frac{1 + 2i}{x + 1 - i}
\end{equation*}
\invisible<2>{
Une primitive de $f$ est donnée par
 \begin{equation*}  
 \begin{split}  
 F(x) & = - (1 + 2i) \left(\frac{1}{2} \ln((x+1)^2 + 1) + i \arctan(x + 1) + k \right)\quad k \in \mathbb{R}
  \\
 & = -\frac{(1 + 2i)}{2} \ln(x^2 + 2x + 2) + (2 - i) \arctan(x + 1) -k(1 + 2i).
 \end{split}
 \end{equation*}
 \invisible<3>{
}}}
 }
\end{frame}
%%%%%%%%
\begin{frame}{Résultat général sur les intégrales de fractions rationnelles}
\invisible<1>{
\begin{proposition}
Soit $f$ la fonction définie sur $\mathbb{R}$ par $   f(x) = \frac{\lambda x + \mu}{x^2 + bx + c}$
où $(\lambda,\mu, b, c)\in \mathbb{R}^{4}$ tel que $b^2 - 4c < 0$. Alors une primitive de $f$ est une combinaison linéaire des fonctions $\dps g : x \mapsto \ln(x^2 + bx + c)$ et $\dps h : x \mapsto \dps \arctan\left(\frac{2x + b}{\sqrt{4c - b^2}} \right)$    
\end{proposition}
\invisible<2>{
\textbf{Démonstration :}
\invisible<3>{
On commence par faire apparaître au numérateur la dérivée du dénominateur. 
\invisible<4>{
    \begin{equation*}
    \begin{split}
        \int_{\mathbb{R}} f(x)\,\mathrm{dx} 
        & = \int_{\mathbb{R}} \frac{\lambda x}{x^2 + bx + c} \,\mathrm{dx} 
        + \int_{\mathbb{R}} \frac{\mu}{x^2 + bx + c} \,\mathrm{dx}
        \\
        & =
    \frac{\lambda}{2}
\int_{\mathbb{R}} \frac{2 x + b}{x^2 + bx + c} \,\mathrm{dx} 
        + \int_{\mathbb{R}} \frac{\mu - b \frac{\lambda}{2}}{x^2 + bx + c} \,\mathrm{dx}
\end{split}        \end{equation*}
\invisible<5>{
}}}}}
\end{frame}
%%
\begin{frame}
Alors
    \begin{equation*}
        \begin{split}
     \int_{\mathbb{R}} f(x)\,\mathrm{dx}    & =     \frac{\lambda}{2}
\int_{\mathbb{R}} \frac{2 x + b}{x^2 + bx + c} \,\mathrm{dx} 
        + (\mu - b \frac{\lambda}{2}) \int \frac{1}{x^2 + bx + c} \,\mathrm{dx}.     
        \end{split}
    \end{equation*}
    \invisible<1>{
    Or
    \begin{equation*}
        \int_{\mathbb{R}} \frac{2 x + b}{x^2 + bx + c} \,\mathrm{dx}  = \ln \myabs{x^2 + bx + c} + k          = \ln (x^2 + bx + c) + k \quad k \in \mathbb{R}.
    \end{equation*}
    \invisible<2>{
    Par ailleurs,
    \begin{equation*}
        \int_{\mathbb{R}} \frac{1}{x^2 + bx + c}\,\mathrm{dx} = 
        \int_{\mathbb{R}} \frac{1}{(x + \frac{b}{2})^2 + \omega^2} = \frac{1}{\omega^2} \int_{\mathbb{R}} \frac{1}{1 + \left(\frac{x + \frac{b}{2}}{\omega}  \right)^2}\,\mathrm{dx} \quad \text{avec} \quad \omega = \sqrt{c - \left(\frac{b}{2}\right)^2}
    \end{equation*}
    \invisible<3>{
    }}}
    \end{frame}
    %%%%%%%%
    \begin{frame}
     \textcolor{byzantine}{Changement de variable :}  $\varphi \in \mathcal{C}^1(\mathbb{R}, \mathbb{R})$ tq $\varphi(u) = \omega u - \frac{b}{2}$ \alert{($u = \frac{x + \frac{b}{2}}{\omega}$, \ $\mathrm{du} = \frac{1}{\omega} \,\mathrm{dx}$)}.
    \begin{equation*}
\int_{\mathbb{R}} \frac{1}{x^2 + bx + c}\,\mathrm{dx} =  \frac{1}{\omega} \int_{\mathbb{R}} \frac{1}{1 + u^2} \, \mathrm{du} = \frac{1}{\omega} \arctan(u) + k_1.
    \end{equation*}
Alors,
\begin{equation*}
\int \frac{1}{x^2 + bx + c}\,\mathrm{dx} = \omega \arctan(\frac{2x + b}{2 \omega}).
    \end{equation*}
    \invisible<1>{
    Finalement
    \begin{equation*}
    \boxed{ \int \frac{\lambda x + \mu}{x^2 + bx + c}\, \mathrm{dx}
         = 
       \frac{\lambda}{2}  \ln(x^2 + bx + c) + 
        \left(\mu - \frac{b \lambda}{2} \right) \sqrt{c - \frac{b^2}{a^2}} \arctan \left( \frac{2x + b}{2 \sqrt{c - \frac{b^2}{a^2}}} \right) + k^{\prime}.}
    \end{equation*}
    \invisible<2>{
    }}
    \end{frame}
    %%%%%%%%%%
    \begin{frame}{Exercice}
    Calculer $\dps \int_{\mathbb{R}} \frac{\mathrm{dx}}{x^3 - 1}$. 
    \vspace*{0.2 cm}
    \\
    \corrige{
    \vspace*{0.2 cm}
    \\
\begin{enumerate}
    \item \textcolor{midnightblue}{
    \textbf{Recherche des pôles}
\begin{equation*}
    x^3 - 1 = (x - 1)(x^2 + x + 1)
\end{equation*}
Alors $1$ est pôle d'ordre 1 mais on ne peut pas trouver d'autres pôles car le polynôme $x^2 + x + 1$ est irréductible.
}
\vspace*{0.2 cm}
\\
\item
\textcolor{midnightblue}{
    \textbf{Décomposition en éléments simples}
Il existe $(\lambda_1, a, b) \in \mathbb{R}^{3}$ tel que
\begin{equation*}
    \frac{1}{x^3 - 1} = \frac{\lambda_1}{x - 1} + \frac{ax + b}{x^2 + x + 1}.
\end{equation*}
Par identification on trouve $a=-\frac{1}{3}$ et $b = \frac{2}{3}$.
}
\end{enumerate}
}
\end{frame}
%%%%%%%
\begin{frame}
\textcolor{midnightblue}{
Alors,
\begin{equation*}
    \frac{1}{x^3 - 1} =
    \frac{1}{3} \frac{1}{x - 1} 
    + \frac{-\frac{1}{3}x + \frac{2}{3}}{x^2 + x + 1}
 =
    \frac{1}{3} \underbrace{\frac{1}{x - 1}}_{A_1} -\frac{1}{6}
     \underbrace{\frac{2x + 1}{(x^2 + x + 1)}}_{A_2} - 5\underbrace{\frac{1}{x^2 + x + 1}}_{A_3}.
\end{equation*}
}
\begin{enumerate}
\setcounter{enumi}{2}
 \item 
 \textcolor{midnightblue}{\textbf{On calcul séparément chaque terme} 
    \begin{align*}
    A_1 = & \int_{\mathbb{R}} \frac{1}{x - 1}\, \mathrm{dx} = \ln \myabs{x - 1} + k_1 \quad k_1 \in \mathbb{R}
    \\
    A_2 = &     \int_{\mathbb{R}} \frac{2x + 1}{x^2 + x + 1}\,\mathrm{dx} = \ln \myabs{x^2 + x + 1} + k_2 \quad k_2 \in \mathbb{R}. 
\end{align*}
De plus
\begin{equation*}
A_3 =  
    \int \frac{1}{(x + \frac{1}{2})^2 + \left(\frac{\sqrt{3}}{2}\right)^2}\,\mathrm{dx}
    = \frac{4}{3} 
    \int_{\mathbb{R}} \frac{1}{  1 + \left( \frac{2 x + 1}{\sqrt{3}} \right)^2}\,\mathrm{dx}
\end{equation*}
}
\end{enumerate}
\end{frame}
%%%%%%%%%
%%%%%%%%%
\begin{frame}
\begin{enumerate}
    \item 
    \textcolor{midnightblue}{\textbf{Changement de variable:}} 
    \textcolor{midnightblue}{
    Soit $\varphi$ définie par $\dps \varphi(u) = \frac{\sqrt{3}}{2} u - \frac{1}{2}$.
Alors
\begin{equation*}
  A_3 = \frac{\sqrt{3}}{2} \int_{\mathbb{R}} \frac{1}{1 + u^2} \,\mathrm{du} 
  = \frac{\sqrt{3}}{2} \arctan(u) + k_3 = \frac{\sqrt{3}}{2}
  \arctan \left(\frac{2x + 1}{\sqrt{3}} \right) + k_3
\end{equation*} 
}
\item
\textcolor{midnightblue}{
\textbf{Conclusion}
\begin{equation*}
\begin{split}
    \int \frac{\mathrm{dx}}{x^3 - 1} 
    & = \frac{1}{3} A_1 - \frac{1}{6} A_2 - 5 A_3
    \\
& =  \frac{1}{3} \ln \myabs{x - 1} -\frac{1}{6} \ln \myabs{x^2 + x + 1} -5 \frac{\sqrt{3}}{2} \arctan \left(\frac{2x + 1}{\sqrt{3}} \right) + k \quad k \in \mathbb{R}.
    \end{split}
\end{equation*}
}
\end{enumerate}
\end{frame}

%%%
%%%%%%%%
\begin{frame}
\frametitle{Règles de Bioche}
\textcolor{cadmiumgreen}{\textbf{But:}}
Trouver le changement de variable optimal pour calculer des intégrales de fractions rationnelles en sinus et cosinus. 
\vspace*{0.2 cm}
\\
Soit $f$ une fraction rationnelle en cosinus et sinus avec $w(x) = f(x)\,\mathrm{dx}$.
\vspace*{0.2 cm}
\\
    \begin{enumerate}
        \item 
        \invisible<1>{
        Si $w(x) = w(-x)$ on définit la fonction $\varphi : [-1, 1] \rightarrow \mathbb{R}$ de classe $\mathcal{C}^1$ sur $]-1,1[$  par 
        \begin{equation*}
 \varphi(u) = \arccos{(u)}       \end{equation*}
        \item
        \invisible<2>{
        Si $\dps w(\pi - x) = w(x)$ on  définit la fonction $\varphi : [-1, 1] \rightarrow \mathbb{R}$ de classe $\mathcal{C}^1$ sur $]-1, 1[$ 
        \begin{equation*}
            \varphi(u) = \arcsin{(u)}.
        \end{equation*}
        \item
        \invisible<3>{
        Si $\dps w(\pi + x) = w(x)$ on définit la fonction $\varphi : \mathbb{R} \rightarrow \left]-\frac{\pi}{2},\frac{\pi}{2} \right[$ de classe $\mathcal{C}^1$ par
        \begin{equation*}
            \varphi(u) = \arctan{(u)}.
        \end{equation*}
        \invisible<4>{
        }}}}
        \end{enumerate}
        \end{frame}
        %%%%%%
        \begin{frame}
\begin{enumerate}
    \setcounter{enumi}{3}
   \item 
   \invisible<1>{
   Si deux des trois propriétés précédentes sont vérifiées 
       on définit la fonction $\varphi : [-1, 1] \rightarrow \mathbb{R}$ de classe $\mathcal{C}^1$ sur $]-1,1[$  par 
        \begin{equation*}
 \varphi(u) = \frac{1}{2} \arccos{(u)}       \end{equation*}
       \item
   \invisible<2>{
   Si aucune des propriétés n'est vérifiée, on utilise le changement de variable "brutal" $\varphi : \mathbb{R} \rightarrow \left]-\frac{\pi}{2},\frac{\pi}{2} \right[$ de classe $\mathcal{C}^1$ par
        \begin{equation*}
            \varphi(u) = 2 \arctan{(u)}.
        \end{equation*}
        \invisible<3>{
        }}}
    \end{enumerate}
\end{frame}
%%%%%%
 \begin{frame}{Exercice}
 Calculer l'intégrale suivante      
 \begin{equation*}
         F(x) = \int \frac{\mathrm{dx}}{\sin{(x)}}
     \end{equation*}
     \begin{enumerate}
         \invisible<1>{
         \item 
        Soit $f$ définie sur $\mathbb{R} \backslash \left\{ k \pi, k \in \mathbb{Z} \right\}$ par $\dps f(x) = \frac{1}{\sin(x)}$ alors  $\dps w(x) = \frac{\mathrm{dx}}{\sin{(x)}} = w(-x)$. 
 \invisible<2>{
 \item
 \textbf{Règles de Bioche} suggèrent de poser $\varphi : [-1, 1] \rightarrow \mathbb{R}$ de classe $\mathcal{C}^1$ sur $]-1, 1[$ par 
 \begin{equation*}
     \varphi(u) = \arccos{(u)}.
 \end{equation*}
 \invisible<3>{
 \item
 \textbf{Formule du changement de variable :}
 \begin{equation*}
 \int \frac{\mathrm{dx}}{\sin(x)} = \int f(\varphi(u)) \varphi^{\prime}(u)\,\mathrm{du} = -\int \frac{1}{\sin(\arccos{(u)})} \frac{1}{\sqrt{1 - u^2}}\,\mathrm{du}.
\end{equation*}
 \invisible<4>{
 }}}}
     \end{enumerate}
     \end{frame}
%     %%%%%%%%%%%%%%
     \begin{frame}
     \begin{enumerate}
     \setcounter{enumi}{3}
         \item 
     Comme $\sin^2(x) + \cos^2(x) = 1$ alors $\dps \sin{(\arccos{(u)})} = \sqrt{1 - u^2}$.
 D'où
 \begin{equation*}
     \int \frac{\mathrm{dx}}{\sin(x)} 
     = -\int \frac{1}{1 - u^2}\,\mathrm{du} = -\int \frac{1}{(1 - u)(1 + u)}\,\mathrm{du}.
 \end{equation*}
 \item
 \textbf{Décomposition en éléments simples :} $\exists ! (\lambda_1, \lambda_2) \in \mathbb{R} \times \mathbb{R}$ tel que
 \begin{equation*}
     \frac{1}{(1 - u)(1 + u)} = \frac{\lambda_1}{1 - u} + \frac{\lambda_2}{1 + u}.
 \end{equation*}
 Après identification on trouve $\dps \lambda_1 = \frac{1}{2}$ et $\dps \lambda_2 = \frac{1}{2}$.
 \item
 \textbf{Conclusion :}
 \begin{equation*}
 \begin{split}
 \int \frac{\mathrm{dx}}{\sin(x)} &=
 -\frac{1}{2} \int \left( \frac{1}{1 - u} + \frac{1}{1 + u} \right)\,\mathrm{du} = \frac{1}{2} \ln{\myabs{\frac{1 - u}{1 + u}}} + \lambda = \frac{1}{2} \ln{\left(\frac{1 - \cos{(x)}}{1 + \cos{(x)}}\right)} + \lambda
 \\
 & = 
 \end{split}
 \end{equation*}
     \end{enumerate}
 \end{frame}
% %%%%%%%%%%%%
 \begin{frame}
     \textcolor{cadmiumgreen}{\textbf{Autre méthode équivalente :}}
     \vspace*{0.4 cm}
     \begin{enumerate}
         \item Règles de Bioche : on pose $u = \cos(x)$ alors $\mathrm{du} = -\sin(x) \,\mathrm{dx}$.
        \vspace*{0.4 cm}
         \item
         Alors
         \begin{equation*}
         \begin{split}
             \int
 \frac{\mathrm{dx}}{\sin(x)} & =\int \frac{\sin(x)}{\sin^2(x)}\,\mathrm{dx} =  \frac{\sin(x)\mathrm{dx}}{1 - \cos^2(x)} = \int -\frac{\mathrm{du}}{1 - u^2} = \frac{1}{2} \ln{\myabs{\frac{1 - u}{1 + u}}} + \lambda
 \\
 & = \frac{1}{2} \ln{\left(\frac{1 - \cos{(x)}}{1 + \cos{(x)}}\right)} + \lambda         \end{split}
 \end{equation*}
     \end{enumerate}
 \end{frame}
