%%%%%%%%%%% CHAPTER FRACTIONS RATIONNELLES%%%%%%%%%

\begin{frame}
\begin{center}
\Huge{Fractions rationnelles}
\end{center}
\end{frame}
   %%%
%%%%%
\begin{frame}{Contenu}
\begin{enumerate}
    \item Rappels et quelques propriétés sur les polynômes
\vspace{0.5 cm}
    \item Définition et propriétés des fractions rationnelles
\vspace{0.5 cm}
    \item Décomposition en éléments simples dans $\mathbb{R}[X]$ et $\mathbb{C}[X]$.
\vspace{0.5 cm}
\item Quelques applications
    
\end{enumerate}
       
\end{frame}

%%%%%%%

\begin{frame}{Polynômes}
 \begin{definition}
      On appelle polynôme à coefficient dans $\mathbb{K}$ toute expression du type
      \begin{equation*}
          A = \sum_{k=0}^n a_k X^k \quad a_k \in \mathbb{R} \ \text{ou} \ \mathbb{C}.
      \end{equation*}
      De plus, le degré du polynôme $A$, noté $\mathrm{deg}(A)$ est définit par
      \begin{equation*}
          \mathrm{deg}(A) = 
          \left\{
\begin{array}{rcr}
& \max \left\{ k \in \mathbb{N} \ | \ a_k \neq 0 \right\} \ & \text{si} \  A \neq 0  \\
& - \infty \ &\text{si} \ A = 0. 
\end{array}
\right.
      \end{equation*}
       $a_n \neq 0$ s'appelle le coefficient dominant du polynôme.
  \end{definition}
    
\end{frame}

%%%%%%%%%%
%%%%%%%%%%%%
\begin{frame}{Quelques propriétés}
\begin{proposition}
Etant donnés deux polynômes $A$ et $B$ de $\mathbb{K}[X]$, on a :
\vspace{0.2 cm}
\begin{enumerate}
    \item $\deg{(A + B)} = \max (\deg (A), \deg (B))$
    \vspace{0.3 cm}
    \item $\deg(AB) = \deg(A) + \deg(B)$
\end{enumerate}
\end{proposition}
%%
\begin{proposition}
    L'ensemble $\mathbb{K}_n[X]$ des polynômes de degré $\leq n$ est un sous-espace vectoriel de $\mathbb{K}[X]$.
\end{proposition}
\textbf{Démonstration :}
\\
\vspace{0.3 cm}
\alert{\textbf{Voir le polycopié !}}
\end{frame}

%%%%%%%%%%%%
%%%%%%%%%%%

\begin{frame}{Divisibilité dans $\mathbb{K}[X]$}
    \begin{definition}
    Soient $(A,B) \in \mathbb{K}[X] \times \mathbb{K}[X]$. On dit que $A$ divise $B$ si
    \begin{equation*}
        \exists C \in \mathbb{K}[X] \quad AC = B
    \end{equation*}
\end{definition}
\invisible<1>{
\textbf{Exercice :} 
Montrer que le polynôme $(X - 1) (X - 2)$ divise le polynôme $(X - 1)^2(X-2) (X^2 + X + 1)$.
\\
\vspace{0.2 cm}
\invisible<2>{
        \corrige{
        \begin{equation*}
            (X - 1)^2(X-2) (X^2 + X + 1) = (X - 1)(X - 2) \times \textcolor{darkmagenta}{R(X)}
        \end{equation*}
        avec 
        \begin{equation*}
\textcolor{darkmagenta}{R(X) = (X - 1) (X^2 + X + 1)}.
        \end{equation*}
        }
        \invisible<3>{
        }}}
\end{frame}
%%%%%%
%%%%%%
\begin{frame}{Divisibilité dans $\mathbb{K}[X]$}
\begin{theorem}
\'{E}tant donnés deux polynômes $A$ et $B$ de $\mathbb{K}[X]$ avec $B \neq 0$, il existe un unique couple $(Q,R) \in \mathbb{K}[X] \times \mathbb{K}[X]$ vérifiant
\begin{equation*}
    A = BQ + R \quad \text{avec} \quad  \deg{(R)} < \deg{(B)}.
\end{equation*}
$Q$ est appelé le quotient et $R$ le reste de la division euclidienne de $A$ par $B$.
\end{theorem}
\vspace{0.3 cm}
    \alert{\textbf{Preuve dans le polycopié! très intéréssante!}}
\end{frame}
%%%%%%%
%%%%%
\begin{frame}{Conjugaison}
\begin{definition}
    Si $\dps A=\sum_{k=0}^{+\infty} a_k X^k \in \mathbb{C}[X]$, on appelle conjugué de $A$, le polynôme $\dps \overline{A} = \sum_{k=0}^{+\infty} \overline{a}_k X^k$.
\end{definition}

\begin{proposition}
Soit $(A, B) \in \mathbb{C}[X]^2$.
Alors, on a les propriétés suivantes : 
\begin{enumerate}
    \item
    $\overline{A + B} = \overline{A} + \overline{B}$
    \item
    $\overline{AB} = \overline{A} \times \overline{B}$
    \item
   $ A \in \mathbb{R}[X] \iff \overline{A} = A$
\end{enumerate}    
\end{proposition}
\alert{\textbf{Preuve dans polycopié!}}
\end{frame}
%%%%
%%%%
\begin{frame}{Fractions rationnelles}
\begin{definition}  
      On appelle corps des fractions rationnelles à coefficients dans $\mathbb{K}$ l'ensemble 
      \begin{equation*}
         \mathcal{K} = \left\{ F = \frac{P}{Q} \quad \text{avec} \quad (P, Q) \in \mathbb{K}[X]^2 \quad \text{et} \quad Q \neq 0 \right\} \quad \textcolor{darkmagenta}{(P,Q) : \ \text{représentant de} \ F}.
      \end{equation*}
       \textcolor{darkmagenta}{Opérations licites :}
      Pour $P$, $P_1$, $Q$, $Q_1$, $R$ des polynômes avec $Q \neq 0$, $Q_1 \neq 0$ et $R \neq 0$ :
      \\
    \begin{minipage}[t]{0.36\linewidth}
    \begin{itemize}
\item [$\bullet$]
$\dps \frac{P}{Q} + \frac{P_1}{Q_1} = \frac{P Q_1 + P_1 Q}{Q Q_1} $
\\
\vspace{0.2 cm}
           \item[$\bullet$]
$\dps \frac{P}{Q} \times \frac{P_1}{Q_1} = \frac{P P_1}{Q Q_1}$
\end{itemize}
\end{minipage}
\begin{minipage}[t]{0.36\linewidth}
\begin{itemize}
           \item[$\bullet$]
$\dps \frac{P}{Q} = \frac{P_1}{Q_1} \iff P Q_1 = P_1 Q$
\\
\vspace{0.2 cm}
           \item[$\bullet$]
$\dps\frac{PR}{QR} = \frac{P}{Q}$
\end{itemize}
\end{minipage}
\begin{minipage}[t]{0.24\linewidth}
\begin{itemize}
           \item[$\bullet$]
 Si $P \neq 0$, $\dps \left( \frac{P}{Q} \right)^{-1} = \frac{Q}{P}$.   
      \end{itemize}
\end{minipage}
  \end{definition}
  \end{frame}
    %%%%
\begin{frame}{Fractions rationnelles}
    \begin{definition}[Conjuguée]
      Soit $F$ une fraction rationnelle à coeff complexes  $\dps F = \frac{P}{Q}$  avec  $(P, Q) \in \mathbb{K}[X]^2$ et $Q \neq 0$. 
      \textcolor{darkmagenta}{Fraction rationnelle conjuguée de $F$ :}
      \begin{equation*}
\overline{F} = \frac{\overline{P}}{\overline{Q}}.
      \end{equation*}
  \end{definition}
    \begin{proposition}
    Soient $F$ et $G$ deux fractions rationnelles.
  Alors
  \begin{equation*}
      \overline{F + G} = \overline{F} + \overline{G} \quad \text{et} \quad \overline{FG} = \overline{F} \overline{G}. 
  \end{equation*}
  \end{proposition}
\end{frame}
%%%%%%%%%
%%%%%%%%%%%
\begin{frame}{Représentant irréductible d'une fraction rationnelle}
\invisible<1>{
     \begin{definition}[Polynômes premiers entre eux]
$A$ et $B$ sont premiers entre eux si les seuls diviseurs communs à $A$ et $B$ sont les polynômes de degré $0$.
  \end{definition}
  \textbf{Exemple :}
  \invisible<2>{
      $P(X) = X \quad Q(X) = (X-1)^3$.
  Le seul diviseur commun à $P$ et $Q$ est le polynôme constant $1$...
\invisible<3>{
  \begin{definition}
On appelle représentant irréductible d'une fraction rationnelle $F$ tout représentant $(P, Q)$ de $F$ où $P$ et $Q$ sont premiers entre eux.    
  \end{definition}
  \invisible<4>{
   \alert{\textbf{Toute fraction rationnelle admet un représentant irréductible unitaire et un seul!}}
   \invisible<5>{
   }}}}}
\end{frame}
%%%%%
%%%%%%%
\begin{frame}{Degré d'une fraction rationnelle}
     \begin{definition}
      Si $F$ est une fraction rationnelle : $\dps F = \frac{P}{Q}$.
      \begin{equation*}
          \deg{(F)} = \deg{(P)} - \deg{(Q)} \in \mathbb{Z} \cup \left\{ - \infty \right\}
      \end{equation*}
  \end{definition}
  \begin{proposition}
  \'{E}tant données deux fractions rationnelles $F_1$ et $F_2$ de $\mathbb{K}[X]$, on a :
  \vspace{0.2 cm}
  \begin{enumerate}
      \item
      $\deg{(F_1 + F_2)} \leq \max (\deg{(F_1)}, \deg{(F_2)})$
\vspace{0.2 cm}
      \item
      $\deg{(F_1 F_2)} = \deg{(F_1)} + \deg{(F_2)}$
  \end{enumerate}
  \end{proposition}
\end{frame}
%%%
\begin{frame}{Démonstration}
   \vspace*{-0.1 cm}
    \begin{enumerate}
        \item 
        Supposons $\deg{(F_1)} \geq \deg{(F_2)}$ 
       \invisible<1>{ 
       \ $\Rightarrow$
        \invisible<2>{ $\textcolor{darkmagenta}{\deg{(P_1)} - \deg{(Q_1)} \geq \deg{(P_2)} - \deg{(Q_2)}}$.
        \invisible<3>{
        \begin{equation*}
            F_1 + F_2 = \frac{P_1 Q_2 + Q_1 P_2}{Q_1 Q_2} \Rightarrow \deg{(F_1 + F_2)} = \deg{(P_1 Q_2 + Q_1 P_2)} - \deg{(Q_1 Q_2)}
            \end{equation*}
            \invisible<4>{
\begin{equation*}
\begin{split}
    \deg{(F_1 + F_2)} & \leq \max(\deg{(P_1 Q_2)}, \deg{(Q_1 P_2)}) - \deg{(Q_1 Q_2)}
    \\
    & = \textcolor{darkmagenta}{\max (\deg{(P_1)} + \deg{(Q_2)}, \deg{(Q_1)} + \deg{(P_2)}) - \deg{(Q_1)} - \deg{(Q_2)}}
    \\
    & = \deg{(P_1)} - \deg{(Q_1)} = \deg{(F_1)} = \max(\deg{(F_1)}, \deg{(F_2)}).
    \end{split}
\end{equation*}   
\invisible<5>{
\item
On a 
\begin{equation*}
\begin{split}
    \deg{(F_1 F_2)} & = \deg{(\frac{P_1}{Q_1} \frac{P_2}{Q_2})} = \deg{(P_1P_2)} - \deg{(Q_1 Q_2)} 
    \\
    & = \deg{(P_1)} + \deg{(P_2)} - \deg{(Q_1)} - \deg{(Q_2)} = \deg{(F_1)} + \deg{(F_2)}
\end{split}
\end{equation*}
\invisible<6>{
    }}}}}}
    \end{enumerate}
\end{frame}
%%
\begin{frame}{Racines et pôles}
\begin{definition}
    Soit $F$ une fraction rationnelle de forme irréductible $\dps \frac{P}{Q}$.
    \vspace{0.2 cm}
    \begin{itemize}
        \item[$\bullet$] On appelle racine de $F$ toute racine de $P$.
        \vspace{0.3 cm}
        \item[$\bullet$] On appelle pôle de $F$ toute racine de $Q$.
        \vspace{0.3 cm}
        \item[$\bullet$] Si $a$ est une racine (respectivement un pôle) de $F \neq 0$, l'ordre de multiplicité de $a$ est l'ordre de multiplicité de $a$ en tant que racine du polynôme $P$ (respectivement $Q$).
    \end{itemize}
\end{definition}    
\end{frame}
%%%
\begin{frame}{Application}
Soit $F$ la fraction rationnelle définie par
    \begin{equation*}
        F = \frac{X^3 - 1}{X^2 - 1}.
    \end{equation*}
Trouver les racines de $F$.
\\
\vspace{0.2 cm}
\corrige{ 
\begin{enumerate}
    \invisible<1>{
    \item \textcolor{darkmagenta}{Ecriture de $F$ sous forme irréductible}
    \begin{equation*}
        F(X) = \frac{(X - 1)(X^2 + X + 1)}{(X + 1)(X - 1)} = \frac{X^2 + X + 1}{X + 1}
    \end{equation*}
    \invisible<2>{
    \item
    Calcul des racines : 
    \begin{equation*}
        X_1 = -\frac{1}{2} - i \frac{\sqrt{3}}{2} \quad X_2 = -\frac{1}{2} + i \frac{\sqrt{3}}{2}
    \end{equation*}
    \invisible<3>{
    }}}
\end{enumerate}
}
\end{frame}
%%%%%
\begin{frame}{Décomposition en éléments simples sur $\mathbb{C}[X]$}
\vspace*{-0.1 cm}
\begin{proposition}[Partie entière]
Toute fraction rationnelle $F$ s'écrit de façon unique comme la somme d'un polynôme, appelé partie entière de $F$, et d'une fraction rationnelle de degré strictement négatif.
\end{proposition}
\invisible<1>{
    \textcolor{midnightblue}{\textbf{Démonstration :}}
    \vspace{0.2 cm}
      \begin{enumerate}
          \item 
\invisible<2>{
\textbf{Existence :} $ F = \frac{P}{Q}$. \textcolor{darkmagenta}{Théorème de la division Euclidienne :} $\exists ! (E, R) \in (\mathbb{K}[X])^2$ tq 
\begin{equation*}
    P = EQ + R \quad \text{avec} \quad \deg{(R)} < \deg{(Q)} \ \Rightarrow   \textcolor{red}{F = E + R / Q}.
\end{equation*}
\invisible<3>
{
\item
          \textbf{Unicité :} Par l'absurde. 
          Supposons  $\exists (E, E_1) \in \mathbb{K}[X]^2$ et $(\Tilde{E}, E_2) \in \mathbb{K}[X]^2$ tq
          \begin{equation*}
              F = E + E_1 \ \text{et} \ F = \Tilde{E} + E_2 \quad \text{avec} \ \deg{(E_1)} < 0 \ \text{et} \ \deg{(E_2)} < 0 \ \Rightarrow  \textcolor{darkmagenta}{E - \Tilde{E} = E_2 - E_1}.
          \end{equation*}
          Si $\deg{(E_1)}> \deg{(E_2)}$, $\deg{(E - \tilde{E})} < 0$,  $\Rightarrow E_1 - E_2 = 0$, c'est à dire $E = \widetilde{E}$.
      \end{enumerate}
      \invisible<4>{
      }}}}
\end{frame}
%%%%
%%%%
\begin{frame}{Application}
\textbf{\textcolor{midnightblue}{Exercice :}}
Calculer la partie entière de la fraction rationnelle 
\begin{equation*}
    F = \frac{X^5}{(X^2 + X + 1)^2}
\end{equation*}
\invisible<1>{
\corrige{
L'algorithme de la division euclidienne de $X^5$ par $(X^2 + X + 1)^2$ donne
\begin{equation*}
    F = \frac{(X - 2)(X^2 + X + 1)^2+ X^3 + 4X^2 + 3X + 2}{(X^2 + X + 1)^2} = \underbrace{X - 2}_{E} + \underbrace{\frac{X^3 + 4X^2 + 3X + 2}{(X^2 + X + 1)^2}}_{\mathrm{deg} < 0}.
\end{equation*}
}    
\invisible<2>{
}}

\end{frame}
%%%%%
%%%%%%%%
\begin{frame}{Partie polaire}
     \begin{proposition}
     Si $F$ est une fraction rationnelle admettant $a$ pour pôle d'ordre $n$, il existe un unique n-uplet de scalaires 
     $(\lambda_p)_{p \in \left[ 1,n \right]}$ et une unique fraction $F_0$ n'admettant pas $a$ pour pôle tq:
  \begin{equation*}
      F(X) = \sum_{p=1}^n \frac{\lambda_p}{(X - a)^p} + F_0.
  \end{equation*}
  La quantité : 
  \begin{equation*}
      \sum_{p=1}^n \frac{\lambda_p}{(X - a)^p}
  \end{equation*}
  s'appelle la partie polaire de $F$ relative au pôle de $a$.
  \end{proposition}
\end{frame}
%%%%%%
%%%%%
\begin{frame}{Méthodologie de détermination des parties polaires}
      Soit $F$ une fraction rationnelle admettant $a$ pour pôle.
       \begin{enumerate}
           \item \textbf{Pôle d'ordre $1$}
  \begin{equation*}
      F(X) = \frac{P(X)}{(X-a) Q_1(X)} \qquad \textcolor{cadmiumgreen}{(P_1)}
  \end{equation*}
  \invisible<1>{
\alert{\textbf{Décomposition en éléments simples :}}  
\begin{equation*}
      F(X) = \frac{\lambda_1}{X-a} + F_0.
  \end{equation*}
  \invisible<2>{
  On multiplie l'équation \textcolor{cadmiumgreen}{$(P_1)$} par $X-a$ et on obtient 
  \begin{equation*}
      \frac{P(X)}{Q_1(X)} = \lambda_1 + (X - a) F_0.
  \end{equation*}
  \invisible<3>{
  Pour $X = a$ on trouve
  \begin{equation*}
      \lambda_1 = \frac{P(a)}{Q_1(a)}.
  \end{equation*}
  La partie relative au pôle $a$ est donc :
  \begin{equation*}
      \frac{\lambda_1}{X - a} \quad \text{avec} \quad \lambda_1 = \frac{P(a)}{Q_1(a)}.
  \end{equation*}
  \invisible<4>{
  }}}}
  \end{enumerate}
       
\end{frame}
%%%%
\begin{frame}{Alternative pratique} 
Le pôle $X_i$ vérifie 
\begin{equation}
\label{eq:autre:moyen}
    \frac{P(X)}{Q(X)} = \frac{P(X)}{(X - X_i) Q_i(X)}.
\end{equation}
Alors, en multipliant~\eqref{eq:autre:moyen} par $X - X_i$ pour $i$ fixé on obtient
\begin{equation*}
\frac{P(X)}{Q_i(X)} = \lambda_i + \sum_{j \neq i} \lambda_j \frac{X - X_i}{X - X_j}.
\end{equation*}
De plus, $Q^{\prime}(X) = Q_i(X) + (X - X_i) Q_i^{\prime}(X)$ et donc $Q_i(X_i) = Q^{\prime}(X_i)$. 
\begin{equation*}
  \textcolor{darkmagenta}{ \boxed{ \lambda_i = \frac{P(X_i)}{Q^{\prime}(X_i)}.}}
\end{equation*}
\end{frame}
%%%%
\begin{frame}
\vspace{-0.3 cm}
    \begin{enumerate} \setcounter{enumi}{1}
        \item \textbf{Pôle d'ordre $2$}
$\dps  F(X) = \frac{P(X)}{(X - a)^2 Q_2(X)}$
\vspace{0.2 cm}
\\
\alert{\textbf{Décomposition en éléments simples :}} 
  $    F(X) = \dps \frac{\lambda_2}{(X - a)^2} + \frac{\lambda_1}{X - a} + F_0$.
  
On multiplie par $(X - a)^2$ :
  \begin{equation*}
      \frac{P(X)}{Q_2(X)} = \lambda_2 + \lambda_1 (X - a) + F_0 (X - a)^2.  \quad \text{Pour} \ X = a \ \Rightarrow \dps \textcolor{darkmagenta}{\lambda_2 = \frac{P(a)}{Q_2(a)}}
  \end{equation*}
On dérive 
  \begin{equation*}
      \left( \frac{P(X)}{Q_2(X)} \right)^{\prime} = \lambda_1 + F_0^{\prime}(X) (X - a)^2 + 2 (X - a) F_0(X) 
  \end{equation*}
  Pour $X = a$ 
  \begin{equation*}
  \textcolor{darkmagenta}{
  \dps \lambda_1 = \left( \frac{P}{Q_2} \right)^{\prime}(a)}      
  \end{equation*}
    \end{enumerate}
\end{frame}
%%%%%
\begin{frame}{Exercice}
          Soit $F = \dps \frac{X^5 + 1}{X(X-1)^2}$.
      Trouvez la partie polaire associée à chaque pôle de $F$.
      \corrige{ 
$-1$ est racine de $F$ donc
\begin{equation*}
 X^5 + 1 = (X + 1)(X^4 - X^3 + X^2 - X + 1). 
\end{equation*}
Finalement, la fraction rationnelle $F$ s'écrit sous la forme
\begin{equation*}
    F(X) = \frac{(X + 1)(X^4 - X^3 + X^2 - X + 1)}{X(X - 1)^2} \quad 0 \ \text{pôle d'ordre 1} \quad 1 \ \text{pôle d'ordre 2}
\end{equation*}
\begin{equation*}
     F(X) = \frac{\lambda_1}{X} + \frac{\lambda_2}{X - 1} + \frac{\lambda_3}{(X - 1)^2} 
\end{equation*}
    Par identification
    \begin{equation*}
          F(X) = \frac{1/2}{X} + \frac{-1/2}{X - 1} + \frac{1/2}{(X - 1)^2}.
      \end{equation*}
      }
\end{frame}
%%
\begin{frame}{Exercice}
Décomposer sur $\mathbb{R}$ la fraction rationnelle suivante :
\begin{equation*}
    F(X) = \frac{X^2 + 2X + 5}{X^2 - 3X + 2}
\end{equation*}
\corrige{
\begin{enumerate}
\item
\invisible<1>{
On écrit \textcolor{darkmagenta}{$F$} comme la \textcolor{darkmagenta}{somme d'une partie entière et d'une fraction rationnelle de degré strictement négatif}. 
\invisible<2>{
Par le \alert{Théorème de la division euclidienne :}
\begin{equation*}
    X^2 + 2X + 5 = 1 \times (X^2 - 3X + 2) + 5X + 3.
\end{equation*}
\invisible<3>{
Donc
\begin{equation*}
    F(X) = 1 + \frac{5X + 3}{X^2 - 3X + 2}
\end{equation*}
\invisible<4>{
}}}}
\end{enumerate}
}
\end{frame}
%%%%
\begin{frame}
    \begin{enumerate}
     \setcounter{enumi}{1}   \item 
        On factorise le dénominateur pour identifier les pôles de $F$.
    On a
    \begin{equation*}
        X^2 - 3X + 2 = (X - 2)(X - 1).
    \end{equation*} 
    \alert{Donc $F$ admet $1$ et $2$ comme pôles d'ordre $1$.}
    \item 
    \invisible<1>{
    \textcolor{darkmagenta}{Théorème de décomposition en éléments simples :} $\exists (\lambda_1, \lambda_2) \in \mathbb{R}^2$ tel que
    \begin{equation*}
        F(x) = 1 + \frac{\lambda_1}{X - 2} + \frac{\lambda_2}{X  - 1}
    \end{equation*}
    \invisible<2>{
    \item Par identification 
    \begin{equation*}
        \lambda_1 = 13  \quad \text{et} \quad \lambda_2 = -8. 
    \end{equation*}
    \item
    \invisible<3>{
    Conclusion :
    \textcolor{darkmagenta}{
    \begin{equation*}
\boxed{        F(X) = 1 + \frac{13}{X - 2} - \frac{8}{X - 1}.
}
\end{equation*}
\invisible<4>{
}}}}
}
    \end{enumerate}
\end{frame}
