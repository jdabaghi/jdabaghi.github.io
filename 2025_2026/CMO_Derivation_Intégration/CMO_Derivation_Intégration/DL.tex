
\miniframesoff
\begin{frame}
\begin{center}   
\Huge{Formules de Taylor}
\end{center}
    \end{frame}
    %%%
    \begin{frame}
    \frametitle{Introduction}
    Les formules de Taylor constituent des outils très intéressants dans l'étude de fonctions.
    Elles permettent
    \vspace*{0.3 cm}
    \begin{enumerate}
    \item
    d'approcher une fonction localement par des polynômes (Taylor--Young)
    \vspace*{0.3 cm}
    \item
d'approcher une fonction globalement par des polynômes et de déduire une expression sur le reste (Taylor reste-intégral)    
\end{enumerate}     
    \end{frame}
    %%%%
    \begin{frame}
    \vspace*{-0.2 cm}
\frametitle{Formules de Taylor}
\begin{theorem}[Formule de Taylor avec reste intégral]
Soient $I$ un intervalle et $a, b \in I$. 
Supposons que $a < b$. 
Si $f \in \mathcal{C}^{n+1}(I)$ alors :
\begin{equation*}
f(b) = \underbrace{\sum_{k=0}^n \frac{(b-a)^k}{k !} f^{(k)}(a)}_{\text{polynôme}} + \underbrace{\int_{a}^{b} \frac{(b-t)^n}{n !} f^{(n+1)}(t)\,\mathrm{dt}}_{\text{reste}}.
\end{equation*}
\end{theorem}
\invisible<1>{
\textcolor{cadmiumgreen}{\textbf{Application :}}
Montrez que $\dps \forall x \in \left[-\pi, \pi \right]$, $\dps \cos(x) \geq 1 - \frac{x^2}{2}$
\\
\vspace*{0.2 cm}
\invisible<2>{
\corrige{Formule de Taylor avec reste intégral à la fonction cos à l'ordre $2$ :
\begin{equation*}
\cos(x) = \sum_{k=0}^2 \frac{x^k}{k !} \cos^{(k)}(0) + \int_{0}^{x} \frac{(x-t)^2}{2}\cos^{(3)}(t)\,\mathrm{dt}
= 1 - \frac{x^2}{2} + \alert{\int_{0}^{x} \frac{(x-t)^2}{2}\sin(t)\,\mathrm{dt} \geq 0}.
\end{equation*}
}
\invisible<3>{
}}}
\end{frame}
%------
\begin{frame}{Formules de Taylor}
    \begin{theorem}[Inégalité de Taylor Lagrange]
Soit $f$ une fonction de classe $\mathcal{C}^{n+1}$ sur $I$. 
Si $M$ majore $|f^{(n+1)}|$ sur le segment $[a, b]$, on a :
\begin{equation*}
\left \lvert f(b) - \sum_{k=0}^n \frac{(b-a)^k}{k !} f^{(k)}(a) \right \rvert \leq M \frac{|b - a|^{n+1}}{(n+1)!}.
\end{equation*}
\alert{formule donne des informations sur l'erreur d'approximation polynomiale !}
\end{theorem}
\invisible<1>{
\textcolor{cadmiumgreen}{\textbf{Exercice :}}
Montrer que $\dps \forall x \in \mathbb{R}, \ \left \lvert \sin(x) - x + \frac{x^3}{6} \right \rvert \leq \frac{x^4}{24}$.
\\
\vspace*{0.2 cm}
\invisible<2>{
}}
\end{frame}
%-----------
%%
\begin{frame}
\corrige{
On applique l'inégalité de Taylor-Lagrange à l'ordre $3$ à la fonction sinus de classe $\mathcal{C}^{\infty}$ sur $\mathbb{R}$ et vérifiant $\forall n \in \mathbb{N}$, $\sup_{x \in \mathbb{R}} \left \lvert f^{(n)}(x) \right \rvert \leq 1$.
\begin{equation*}
\left \lvert \sin(x) - \sum_{k=0}^3 \frac{x^k}{k!} f^{(k)}(0) \right \rvert = \left \lvert \sin(x) - x  - \frac{x^3}{6} \right \rvert \leq \frac{|x|^{4}}{4 !} = \frac{x^{4}}{24} . 
\end{equation*}
}
\end{frame}

\begin{frame}{Formules de Taylor}
    \begin{theorem}[Formule de Taylor-Young]
Si $f$ est une fonction de classe $\mathcal{C}^{n}$ sur $I$, il existe une fonction $\varepsilon$ définie sur $I$ telle que : 
\begin{equation*}
\begin{split}
\forall x \in I, \ f(x) & = \sum_{k=0}^n \frac{(x - a)^k}{k !} f^{(k)}(a) + (x-a)^n \varepsilon(x) \quad \text{avec} \quad \lim_{x \rightarrow a} \varepsilon(x) = 0.
\\
\iff f(x) & = \sum_{k=0}^n \frac{(x - a)^k}{k !} f^{(k)}(a) + o((x-a)^n)
\end{split}
\end{equation*}
\end{theorem}
\alert{Formule très importante ! Elle permet de déterminer le développement limité de $f$ à l'ordre $n$.}
\begin{center}
\textcolor{midnightblue}{Mais... peu commode en pratique...}    
\end{center}


\end{frame}
%---
\begin{frame}{Applications}
\begin{enumerate}
    \item<1-> Développement limité de $x \mapsto e^x$ au voisinage de $0$.
    
    \corrige{La fonction $x \mapsto e^x$ est de classe $\mathcal{C}^{\infty}$. La formule de Taylor-Young donne
    \begin{equation*}
        e^x = 1 + x + \frac{x^2}{2!} + \frac{x^3}{3!} + \cdots + \frac{x^n}{n!} + o(x^n) 
    \end{equation*}
    }
    \item<2-> Développement limité de $x \mapsto \cos(x)$ au voisinage de $0$. 
    
    \corrige{
    La fonction $x \mapsto \cos(x)$ est de classe $\mathcal{C}^{\infty}$. La formule de Taylor-Young donne
    \begin{equation*}
    \begin{split}
        cos(x) &= 1 + \frac{x}{1!} \cos^{\prime}(0) + \frac{x^2}{2!} \cos^{(2)}(0) + \frac{x^3}{3!} \cos^{(3)}(0) + \frac{x^4}{4!} \cos^{(4)}(0) + \cdots + o(x^n)
        \\
        & = 1 - \frac{x^2}{2} + \frac{x^4}{4!} + \cdots + (-1)^{n} \frac{x^{2n}}{2n!} + o(x^{2n})
        \end{split}
    \end{equation*}
    }
\end{enumerate}
\end{frame}
%-------
\begin{frame}
\begin{enumerate}
\setcounter{enumi}{2}
    \item<1-> Développement limité de $x \mapsto \sin(x)$ au voisinage de $0$.

    \corrige{La fonction $x \mapsto \sin(x) \in \mathcal{C}^{\infty}$. 
    La formule de Taylor-Young donne
    \begin{equation*}
    \begin{split}
        \sin(x) &= \frac{x}{1!} \sin^{\prime}(0) + \frac{x^2}{2!} \sin^{(2)}(0) + \frac{x^3}{3!} \sin^{(3)}(0) + \cdots + \frac{x^n}{n!} \sin^{(n)}(0) + o(x^n) 
        \\
        & = x - \frac{x^3}{3!} + \frac{x^5}{5!}+ \cdots (-1)^{n} o(x^{2n+1})
        \end{split}
    \end{equation*}
    }
    \item<2->
    Développement limité de $(1+x)^{\alpha}$ où $x > -1$ et $\alpha \in \mathbb{R}$.

    \corrige{La fonction $x \mapsto (1 + x)^{\alpha}$ est $\mathcal{C}^{\infty}$ sur $]-1, + \infty[$. La formule de Taylor-Young donne
    \begin{equation*}
(1 + x)^{\alpha} = 1 + \alpha x + \alpha(\alpha - 1)\frac{x^2}{2} + \cdots + \alpha (\alpha - 1)  \cdots (\alpha - n + 1) \frac{x^n}{n!} + o(x^n)
    \end{equation*}
    }
\end{enumerate}
    
\end{frame}


\begin{frame}
\begin{center}
\Huge{Développements limités}
\end{center}
   
\end{frame}

%---------------------------------------------------------

%------------------------------------------------

%------------------------------------------------

\miniframesoff
\begin{frame}
\frametitle{Développements limités}
\begin{definition}
Une fonction $f$ admet un développement limité  l'ordre $n$ au voisinage de $0$ s'il existe des réels $a_0$, $a_1$, $\cdots$, $a_n$ et une fonction $\varepsilon$ définie sur $\mathcal{D}_f$ tels que : 
\begin{equation*}
\forall x \in \mathcal{D}_f, \ f(x) = \textcolor{red}{\underbrace{\sum_{k=0}^n a_k x^k}_{\text{Partie régulière}}} + \underbrace{x^n \varepsilon(x)}_{\text{\textcolor{midnightblue}{Reste}}} \quad \text{avec} \quad \lim_{x \rightarrow 0} \varepsilon(x) = 0.
\end{equation*}
\end{definition}
\textcolor{cadmiumgreen}{\textbf{Remarque :}}
\'{E}criture équivalente : 
\begin{equation*}
f(x) = \sum_{k=0}^n a_k x^k  + o(x^n).
\end{equation*}
\end{frame}
%%
\begin{frame}
  \frametitle{Quelques exemples}
  
\begin{enumerate}
\item
$f : ]-1, 1[ \rightarrow \mathbb{R}$ définie par
\begin{equation*}
f(x) = x - x^2 + 2x^3 + x^3 \ln(1 + x)
\end{equation*}
admet un DL à l'ordre $3$ en $0$ car 
\begin{equation*}
  \forall x \in ]-1, 1[, \ f(x) = \alert{x-x^2+2x^3} + \textcolor{midnightblue}{x^3 \varepsilon(x)} \quad \text{avec} \quad \varepsilon(x) = \ln(1+x) \underset{0}{\rightarrow} 0
\end{equation*}
\invisible<1>{
\item 
Si une fonction $f$ est de classe $\mathcal{C}^n$ sur un intervalle contenant $0$, alors la formule de Taylor-Young prouve qu'elle admet un développement limité à l'ordre $n$ en $0$ qui s'écrit :
\begin{equation*}
f(x) = \alert{\sum_{k=0}^{n} \frac{f^{(k)}(0)}{k !} x^k} + \textcolor{midnightblue}{o(x^n)}.
\end{equation*}
\invisible<2>{
  }}
\end{enumerate}

\end{frame}
%%
%%
\begin{frame}
\begin{proposition}[Unicité du DL]
Si $f$ est une fonction pour laquelle il existe deux $(n + 1)$-listes de réels $(a_0, a_1,\cdots,a_n)$ et $(b_0, b_1, \cdots, b_n)$ vérifiant~: 
\begin{equation*}
f(x) = \sum_{k=0}^n a_k x^k + o(x^n) \quad \text{et} \quad f(x) = \sum_{k=0}^n b_k x^k + o(x^n),
\end{equation*}
alors
\begin{equation*}
(a_0, a_1,\cdots,a_n) = (b_0, b_1, \cdots, b_n).
\end{equation*}

\end{proposition}
  
\end{frame}
%%%
\begin{frame}
\frametitle{Parité et développements limités}
\begin{proposition}
  Si $f$ admet en $0$ un DL à l'ordre $n$ dont la partie régulière est $P(x) = \sum_{k=0}^n a_k x^k$.
 \begin{itemize}
 \item
Si $f$ est paire, alors $P(x)$ ne contient que des puissances paires de $x$.
 \item
 Si $f$ est impaire, alors $P(x)$ ne contient que des puissances impaires de $x$.
 \end{itemize}
\end{proposition}
\invisible<1>{
\textbf{Démonstration :}
   $f$ admet un \textbf{DL} à l'ordre $n$ en $0$ donc $f(x) = \sum_{k=0}^n a_k x^k + o(x^n)$
   \\
   \textcolor{cadmiumgreen}{\textbf{$f$ est paire :}} 
$\forall x \in \mathcal{V}_{0}, \ f(x) = f(-x) = \dps \sum_{k=0}^n a_k (-1)^k x^k + o((-x)^n)$.
\\
\textcolor{cadmiumgreen}{\textbf{Unicité du DL: }} $\forall 1 \leq k \leq n, \ a_k (-1)^k = a_k$.
\begin{center}
 le polynôme $P$ ne contient que des puissances paires de $x$.
\end{center}
\invisible<2>{
  }}

\end{frame}
%%%%%%
\begin{frame}
\frametitle{Développements limités en $0$ des fonctions élémentaires}
\textcolor{cadmiumgreen}{\textbf{Fonction exponentielle :}}
$
\dps e^x = 1 + \frac{x}{1 !} + \frac{x^2}{2 !} +  \cdots + \frac{x^n}{n !} + o(x^n)$
  \begin{figure}[H]
  \centering
  \includegraphics[width = 0.47 \textwidth]{DL_exp}
  \end{figure}
  \end{frame}
  %%
  \begin{frame}
  \vspace*{-0.2 cm}
\textcolor{cadmiumgreen}{\textbf{La fonction hyperbolique $\ch$ :}} $\dps \ch(x) = 1 + \frac{x^2}{2 !} + \frac{x^4}{4 !} + \cdots + \frac{x^{2n}}{(2n) !} + o(x^{2n + 1})$
 
   \begin{figure}
    \centering
\includegraphics[width = 0.5 \textwidth]{DL_ch}
    \end{figure}
    \end{frame}
    %%%
    \begin{frame}
\vspace*{-0.2 cm}    
    \textcolor{cadmiumgreen}{\textbf{La fonction hyperbolique $\sh$ :}}
 $\dps \sh(x) = x + \frac{x^3}{3 !} + \frac{x^5}{5 !} +  \cdots + \frac{x^{2n+1}}{(2n+1) !} + o(x^{2n+2})$
    \begin{figure}
    \centering
    \includegraphics[width = 0.5 \textwidth]{DL_sh}
    \end{figure}
    \end{frame}
    %%%
    \begin{frame}
\textcolor{cadmiumgreen}{\textbf{La fonction sinus :}}
  $
   \dps \sin(x) = x - \frac{x^3}{3!} + \frac{x^5}{5!} + \cdots + (-1)^n \frac{x^{2n+1}}{(2n + 1)!} + o(x^{2n+2})$
  \begin{figure}[H]
  \centering
  \includegraphics[width = 0.45 \textwidth]{DL_sinus}
  \end{figure}
  \end{frame}
  %%%
  \begin{frame}
  \textcolor{cadmiumgreen}{\textbf{La fonction cosinus :}}
 $ \dps \cos(x) = 1 - \frac{x^2}{2!} + \frac{x^4}{4!} + \cdots + (-1)^n \frac{x^{2n}}{(2n)!} + o(x^{2n+2})$
  \begin{figure}[H]
  \centering
  \includegraphics[width = 0.53 \textwidth]{DL_cos}
  \end{figure}
  \end{frame}
  \begin{frame}
  \textcolor{cadmiumgreen}{\textbf{La fonction $x \mapsto \ln(1+x)$ :}}
  \begin{equation*}
\ln(1+x) = x - \frac{x^2}{2} + \frac{x^3}{3} + \cdots + (-1)^{n-1} \frac{x^n}{n} + o(x^n) 
  \end{equation*}
\textcolor{cadmiumgreen}{\textbf{Pour $\alpha$ un réel quelconque, la fonction $x \mapsto (1+x)^{\alpha}$ :}}
  \begin{equation*}
(1 + x)^{\alpha} = 1 + \alpha x + \frac{\alpha (\alpha - 1)}{2 !} x^2 + \cdots + \frac{\alpha (\alpha - 1) \cdots (\alpha - n + 1)}{n !} x^n + o(x^n)
  \end{equation*}
\textcolor{cadmiumgreen}{\textbf{La fonction $\dps x \mapsto \frac{1}{1-x}$ :}} 
  \begin{equation*}
\frac{1}{1-x} = \sum_{k=0}^{n} x^k + o(x^n)
    \end{equation*}
\end{frame}
%%%
\begin{frame}
\frametitle{Application}
   Déterminer le DL à l'ordre $3$ au voisinage de $2$ de $f$ définie sur $\mathbb{R}^{*}$ par $\dps f(x) = \frac{1}{x}$.
\corrige{
\vspace*{0.3 cm}
\begin{enumerate}
\invisible<1>{
\item
  \textcolor{midnightblue}{Transformation de l'expression
\vspace*{0.2 cm}
  \begin{equation*}
 f(x) =  \frac{1}{2 + x - 2} = \frac{1}{2} \frac{1}{ \left( 1 + \dps \frac{x-2}{2}  \right)}.
    \end{equation*}}
  \vspace*{0.3 cm}
  \invisible<2>{
\item
  \textcolor{midnightblue}{ Changement de variable.
        On pose $h = x - 2$. Alors $h$ tend vers $0$ au voisinage de $2$.
  \begin{equation*}
f(x) = \dps  \frac{1}{2} \frac{1}{1 + \dps \frac{h}{2}} \qquad \text{\textcolor{cadmiumgreen}{DL de} }  \ \ \textcolor{cadmiumgreen}{\frac{1}{1+u}} \quad \text{\textcolor{cadmiumgreen}{en 0 !}}
  \end{equation*}
  }
 \invisible<3>{
   }}}
\end{enumerate}
}
\end{frame}
%%%
\begin{frame}
  \begin{enumerate}
    \setcounter{enumi}{2}
\invisible<1>{
  \item
     \textcolor{midnightblue}{DL en $0$ de $\dps u \mapsto \frac{1}{1 + u}$
       \begin{equation*}
 \dps \frac{1}{1 + u} =  1 - u + u^2 - u^3 + o(u^3).
         \end{equation*}
     }
     \invisible<2>{
   \item
  \textcolor{midnightblue}{On remplace $u$ par $\dps \frac{h}{2} \rightarrow 0$ : 
 \begin{equation*}
 \dps \frac{1}{2} \left( \frac{1}{1 + \frac{h}{2}} \right) = \frac{1}{2} \left( 1 - \frac{h}{2} + \frac{h^2}{4} - \frac{h^3}{8} + o \left(\frac{h^3}{8}\right)\right).
 \end{equation*}       
  }
  \invisible<3>{
         \item
  \textcolor{midnightblue}{ DL de $f$ au voisinage de $x=2$ (\textcolor{cadmiumgreen}{$h = x - 2 \rightarrow 0$}) :
 \begin{equation*}
 f(x) = \frac{1}{2} - \frac{1}{4}(x-2) + \frac{1}{8} (x - 2)^2 - \frac{1}{16} (x - 2)^3 + \frac{1}{2} o \left(\left(\frac{x - 2}{2}\right)^{3}\right).
 \end{equation*}
  }
  \invisible<4>{
    }}}}
 \end{enumerate}
 \end{frame}
%%%%%
\begin{frame}
  \begin{enumerate}
    \setcounter{enumi}{5}
\invisible<1>{
  \item
    \textcolor{midnightblue}{
Simplification des termes négligeables :
\begin{equation*}
o \left(\left(\frac{x - 2}{2}\right)^{3}\right) = o((x-2)^3)
\end{equation*}
car
\begin{equation*}
\begin{split}
o \left(\left(\frac{x - 2}{2}\right)^{3}\right) &= \left(\frac{x - 2}{2}\right)^{3} \varepsilon  \left(\frac{x - 2}{2}\right) \quad \text{avec} \quad \lim_{x \rightarrow 2} \varepsilon \left( \frac{x - 2}{2} \right) = 0
\\
& = (x - 2)^3 \varepsilon_1(x) \quad \text{avec} \quad  \varepsilon_1(x) = \frac{1}{8} \varepsilon \left( \frac{x - 2}{2} \right) \quad \text{où} \quad \lim_{x \rightarrow 2} \varepsilon_1(x) = 0
\\
& = o((x - 2)^3)
\end{split}
\end{equation*}
    }
    \invisible<2>{
  \item
    \textcolor{midnightblue}{ Conclusion :
 \begin{equation*}
   f(x) = \frac{1}{2} - \frac{1}{4}(x-2) + \frac{1}{8} (x - 2)^2 - \frac{1}{16} (x - 2)^3 + o \left((x - 2)^3\right).
 \end{equation*}
    }
    \invisible<3>{
      }}}
\end{enumerate}
\end{frame}
%%%%
\begin{frame}
\frametitle{Dérivabilité et développement limité}

%Si $f$ admet une limite réelle $\ell$ en $x_0 \in \mathbb{R}$ alors $f$ possède un DL à l'ordre $0$ en $x_0$. En effet,
%\begin{equation*}
% f(x) = \ell + \underbrace{(f(x) - \ell)}_{\varepsilon(x)} \quad \text{avec} \quad \lim_{x \rightarrow x_0} \varepsilon(x) = 0.
%  \end{equation*}
%\textbf{Réciproquement.} Si $f$ admet un DL à l'ordre $0$ en $x_0 \in \mathbb{R}$, alors
%\begin{equation*}
%f(x) = a_0 + \varepsilon(x) \quad \text{avec} \quad \lim_{x \rightarrow x_0} \varepsilon(x) = 0.
%  \end{equation*}
%Alors $f$ possède en $x_0$ une limite égale à $a_0$.
%\begin{itemize}
%  \item[$\bullet$]
%    De plus, si $x_0 \in \mathcal{D}_f$, la fonction $f$ est continue en $x_0$.
%    En effet, $\lim_{x \rightarrow x_0} f(x) = a_0$ et $f(x_0) = a_0$ donc $\lim_{x \rightarrow x_0} f(x) = f(x_0)$.
%    
%  \item[$\bullet$]
%Si $x_0 \not \in \mathcal{D}_f$, on peut prolonger $f$ par continuité en posant $f(x_0) = a_0$.
%    
%\end{itemize}
%Le résultat suivant est très utile en pratique :
\begin{proposition}
Soit $f$ une fonction définie sur $\mathcal{D}_f$.
  Alors $f$ est continue en $x_0$ si, et seulement si, $f$ admet un DL à l'ordre $0$ en $x_0$.
  Précisément, dans ce cas, au voisinage de $x_0$
  \begin{equation*}
f(x) = f(x_0) + o(1).
  \end{equation*}
\end{proposition}
\invisible<1>{
\textbf{Démonstration :}
  $\textcolor{red}{(\Rightarrow)}$ Si $f$ est continue en $x_0$ : $\lim_{x \rightarrow x_0} f(x) = f(x_0)$.

  On définit $\varepsilon$ par $\varepsilon(x) = f(x) - f(x_0)$. Alors $\lim_{x \rightarrow 0} \varepsilon(x) = 0$.
  Ainsi, au voisinage de $x_0$
  \begin{equation*}
f(x) = f(x_0) + x^0 \times \varepsilon(x) \quad \text{avec} \quad \lim_{x \rightarrow x_0} \varepsilon(x) = 0 \Rightarrow f(x) = f(x_0) + o(1).
  \end{equation*}
  \invisible<2>{
  $\textcolor{red}{(\Leftarrow)}$ si $f$ admet un DL à l'ordre $0$ en $x_0$: $f(x) = a_0 + \varepsilon(x)$ où $\lim_{x \rightarrow x_0} = \varepsilon(x) = 0$.
  Alors
  \begin{equation*}
  \lim_{x \rightarrow x_0} f(x) = a_0 \quad \Rightarrow \quad \text{f continue en} \ x_0
  \end{equation*}
  \invisible<3>{
    }}}
\end{frame}
%%%
\begin{frame}
\vspace*{-0.2 cm}
\begin{proposition}
    $f$ est dérivable en $x_0$ si, et seulement si, $f$ possède un DL à l'ordre $1$ en $x_0$. 
    Dans ce cas :
    \begin{equation*}
\lim_{x \rightarrow x_0} f(x) = f(x_0) + f^{\prime}(x_0) (x - x_0) + o(x - x_0).
      \end{equation*}
\end{proposition}
\invisible<1>{
\textbf{Démonstration :}
$\textcolor{red}{(\Rightarrow)}$
 Si $f$ est dérivable en $x_0$ alors $\lim_{x \rightarrow x_0} \frac{f(x) - f(x_0)}{x - x_0} = f^{\prime}(x_0).$
\\
On pose :
\begin{equation*}
\varepsilon(x) = \frac{f(x) - f(x_0)}{x - x_0} - f^{\prime}(x_0) \quad \text{si} \quad x \in \mathcal{D}_f \backslash \left\{x_0\right\} \quad \text{et} \quad \lim_{x \rightarrow x_0} \varepsilon(x) = 0.
\end{equation*}
On a $f(x) = f(x_0) + (x - x_0) f^{\prime}(x_0) + (x - x_0) \varepsilon(x)$. Alors, $f$ admet un DL à l'ordre $1$ en $x_0$.
\\
\invisible<2>{
$\textcolor{red}{(\Leftarrow)}$ 
 Si $f$ admet un DL à l'ordre $1$ en $x_0$ :
\begin{equation*}
f(x) = a_0 + a_1 (x - x_0) + o(x - x_0) \Rightarrow \lim_{x \rightarrow x_0} \frac{f(x) - f(x_0)}{x- x_0} = a_1.
\end{equation*}
alors $f$ est dérivable en $x_0$ et $f^{\prime}(x_0)=a_1$.
\invisible<3>{
  }}}
\end{frame}
%%%%%%
\begin{frame}
\frametitle{Opérations sur les développements limités}
\textcolor{cadmiumgreen}{\textbf{Remarque :}}
\\
La formule de Taylor-Young permet de calculer le DL d'une fonction en un point.
%\\\\
\begin{center}
\textcolor{red}{Pas toujours le bon choix!}
\end{center}
\textcolor{cadmiumgreen}{\textbf{Exemple:}}
  DL à l'ordre $5$ au voisinage de $0$ de
\begin{equation*}
 f(x) = \sin(x) e^{x} \frac{1}{\sqrt{1 + x}}
 \end{equation*}
 \alert{Calcul des dérivées successives très coûteux!}
\\
\vspace*{0.2 cm}
 \textcolor{cadmiumgreen}{\textbf{Alternative :} }
 Opérations élémentaires pour calculer des DL
 \begin{enumerate}
 \item
 somme
 \item
produit, quotient
\item
 composition
 \end{enumerate}
\end{frame}
%%%
\begin{frame}
\frametitle{Somme de Développement limités}
  \begin{proposition}
    Soient $f$ et $g$ deux applications de $\mathcal{D}$ dans $\mathbb{R}$ admettant en $0$ des DL à l'ordre $n$ :
    \begin{equation*}
f(x) = P(x) + o(x^n) \quad \text{et} \quad g(x) = Q(x) + o(x^n).
    \end{equation*}
    Alors, le \textbf{DL de $f+g$} en $0$ est : $f(x) + g(x)  =  P(x) + Q(x) + o(x^n)$.  
  \end{proposition}
  \invisible<1>{
\textbf{Démonstration :}
Il existe des fonctions $\varepsilon_1$ et $\varepsilon_2$ définies sur $\mathcal{D}$ telles que :
    \begin{eqnarray*}
      \forall  x \in \mathcal{V}_0, \ f(x) &=& P(x) + x^n \varepsilon_1(x) \quad \text{avec} \quad \lim_{x \rightarrow 0} \varepsilon_1(x) = 0
      \\
      \forall  x \in \mathcal{V}_0, \ g(x) &=& Q(x) + x^n \varepsilon_2(x) \quad \text{avec} \quad \lim_{x \rightarrow 0} \varepsilon_2(x) = 0.
    \end{eqnarray*}
    $\Rightarrow$ $
    \forall x \in \mathcal{V}_0, \ f(x) + g(x) = P(x) + Q(x) + x^n \varepsilon(x) \quad \text{où} \quad \varepsilon(x) = \varepsilon_1(x) + \varepsilon_2(x) \rightarrow 0$.
    \invisible<2>{
      }}
\end{frame}
%%%%
\begin{frame}
\frametitle{Application}
      Déterminer le développement limité à l'ordre $3$ au voisinage de $0$ de la fonction $f$ définie sur $\mathbb{R} \backslash \left\{1 \right\}$ par $\dps f(x) = \frac{1}{1 - x} - e^{x}$.
      \\
      \vspace*{0.2 cm}
      \invisible<1>{
      \corrige{
              Au voisinage de $0$
      \begin{equation*}
\frac{1}{1 - x} = 1 + x + x^2 + x^3 + o(x^3)
      \end{equation*}
      et
      \begin{equation*}
e^x = 1 + x + \frac{x^2}{2} + \frac{x^3}{6} + o(x^3).
      \end{equation*}
          Par somme de développements limités on obtient au voisinage de $0$
    \begin{equation*}
f(x) = -\frac{1}{2}x^2 + \frac{5}{6} x^3 + o(x^3).
    \end{equation*}
      }
      \invisible<2>{
        }}
\end{frame}
\begin{frame}
\frametitle{Produit de Développements limités}

  \begin{proposition}
    Soient $f$ et $g$ deux applications de $\mathcal{D}$ dans $\mathbb{R}$ admettant en $0$ des DL à l'ordre $n$ :
    \begin{equation*}
f(x) = P(x) + o(x^n) \quad \text{et} \quad g(x) = Q(x) + o(x^n).
    \end{equation*}
    Alors, la fonction $fg$ admet au voisinage de $0$ un DL à l'ordre $n$ qui s'écrit :
    \begin{equation*}
      f(x) g(x) = R(x) + o(x^n)
    \end{equation*}
    \textbf{où $R$ est le polynôme obtenu en ne gardant, dans le produit $PQ$, que les termes de degré inférieur ou égal à $n$.}
  \end{proposition}

\end{frame}
%%%%
\begin{frame}
\textbf{Démonstration:}
    \begin{eqnarray*}
       \ f(x)  g(x) &=& \left(P(x) + x^n \varepsilon_1(x) \right) \left(Q(x) + x^n \varepsilon_2(x) \right)  \\
      & = & P(x) Q(x) + x^n \left( \varepsilon_1(x) Q(x) + \varepsilon_2(x) P(x) + x^n \varepsilon_1(x) \varepsilon_2(x) \right).
    \end{eqnarray*}
    Soit $R$ le polynôme obtenu en ne gardant dans le produit $PQ$ que les termes de degré inférieur ou égal à $n$.
    Alors
    \begin{equation*}
\forall x \in \mathcal{V}_0, \ P(x) Q(x) = R(x) + x^{n+1} T(x) \quad \text{où} \quad \mathrm{deg}(T) \leq n - 1.
    \end{equation*}
    Donc
    \begin{equation*}
      \forall x \in \mathcal{V}_0, \ f(x) g(x) = R(x) + x^{n+1} T(x) + x^n (\underbrace{\varepsilon_1(x) Q(x) + \varepsilon_2(x) P(x) + x^n \varepsilon_1(x) \varepsilon_2(x)}_{\alert{=\varepsilon(x) \rightarrow 0}} )
    \end{equation*}
\begin{center}
Ainsi, $fg$ admet $R$ comme DL à l'ordre $n$ au voisinage de $0$.
\end{center}
\end{frame}
%%%%%%%
\begin{frame}
\frametitle{Application}
Déterminer le DL à l'ordre $3$ en $0$ de $g$ définie sur $]-1, + \infty[$ par $g(x) = \dps \frac{\cos (x)}{\sqrt{1+x}}$.
    \corrige{
      }
\begin{enumerate}
 
  \item
  \textcolor{midnightblue}{   \textcolor{cadmiumgreen}{\textbf{DL en 0 de $x \mapsto \cos(x)$ à l'ordre $3$}}
    }
  \invisible<1>{
    \textcolor{midnightblue}{
    \begin{equation*}
\cos(x) = 1 - \frac{x^2}{2} + o (x^{3}) = P(x) + o(x^{3}) 
    \end{equation*}
    }
    \invisible<2>{
      \vspace*{-0.2 cm}
    \item
\textcolor{midnightblue}{
      \textcolor{cadmiumgreen}{\textbf{DL en 0 de $x \mapsto \frac{1}{1 + x}$}}
    \begin{equation*}
      \frac{1}{\sqrt{1 + x}} = (1 + x)^{-\frac{1}{2}} = 1 - \frac{1}{2} x + \frac{3}{8} x^2 -\frac{5}{16} x^3 + o(x^3) = Q(x) + o(x^3)
    \end{equation*}
    }
\invisible<3>{
  \vspace*{-0.2 cm}
    \item
    \textcolor{midnightblue}{DL de $g$ obtenu en ne gardant dans le produit que les termes de degré $\leq 3$.
    \begin{equation*}
g(x) = 1 -\frac{1}{2}x - \frac{1}{8}x^2 - \frac{1}{16}x^3  + o(x^3).
      \end{equation*}
    }
\invisible<4>{
}}}}
\end{enumerate}
\end{frame}
%%%
\begin{frame}
\frametitle{Application}
      Déterminer le DL à l'ordre $4$ en $0$ de la fonction $g$ définie sur $]-\frac{\pi}{2}, \frac{\pi}{2}[$ par $\dps g(x) = \frac{1}{\cos(x)}$

\corrige{
      On a
      \begin{equation*}
g(x) = \frac{1}{\cos(x)} = \frac{1}{1 - u(x)} \quad \text{où} \quad u(x) = 1 - \cos(x) \underset{x \rightarrow 1}{\rightarrow} 0. 
      \end{equation*}
      }
      \begin{enumerate}
\invisible<1>{
      \item
     \textcolor{midnightblue}{$u \in \mathcal{C}^{\infty}(]-\frac{\pi}{2}, \frac{\pi}{2}[)$ donc par Taylor--Young, $u$ admet un DL à l'ordre $4$ en $0$.}
     \vspace*{0.2 cm}
     \invisible<2>{
      \item
     \textcolor{midnightblue}{ La fonction $x \mapsto \cos(x)$ admet en $0$ le DL à l'ordre $4$ :
      \begin{equation*}
\cos(x) = 1 - \frac{x^2}{2} + \frac{x^4}{24} + o(x^4) 
      \end{equation*}
      Donc
     \begin{equation*}
 1 - \cos(x) = \frac{x^2}{2} - \frac{x^4}{24} - o(x^4)
     \end{equation*}
     }
     \invisible<3>{
       }}}
      \end{enumerate}
      \end{frame}
      %%%
      \begin{frame}
      \begin{enumerate}
      \setcounter{enumi}{2}
\invisible<1>{
    \item
\textcolor{midnightblue}{la fonction $\dps u \mapsto \frac{1}{1 - u}$ admet le DL à l'ordre $4$ en $0$:
      \begin{equation*}
\frac{1}{1 - u} = 1 + u + u^2 + u^3 + u^4 + o(u^4).
      \end{equation*}
      }
\invisible<2>{
\item
  \textcolor{midnightblue}{
       On utilise la règle du produit de DL :
      \begin{equation*}
\left( 1 - \cos(x) \right)^2 = \frac{x^4}{4} + o(x^4)
        \end{equation*}      
       D'où
      \begin{equation*}
g(x) = 1 + \frac{x^2}{2} + \frac{5x^4}{24}  + o(x^4).
      \end{equation*}
  }
  \invisible<3>{
    }}}
            \end{enumerate}     
      \end{frame}
      %%%%
      \begin{frame}
      \frametitle{Intégration des développements limités}
       \begin{proposition}
         Soit $I$ un intervalle contenant $0$ et $f : I \rightarrow \mathbb{R}$ une fonction continue possédant en $0$ un DL à l'ordre $n$ qui vaut $\sum_{k=0}^na_k x^k$.
         Si $F$ est une primitive de $f$, alors elle admet un DL à l'ordre $n+1$ en $0$ qui est :
    \begin{equation*}
F(0) + \sum_{k=0}^n \frac{a_k}{k+1}x^{k+1}.
    \end{equation*}
       \end{proposition}
       \textbf{Remarque: } Très pratique pour retrouver le DL d'une fonction dont on connait la primitive ($x \mapsto \arctan(x)$, $x \mapsto \ln(1 + x)$, etc...).
      \end{frame}
      %%
      \begin{frame}
        \frametitle{Applications}
          Ecrivons le DL à l'ordre $n$ de $\dps \frac{1}{1+x}$ :

          \corrige{
          \begin{equation*}
            \frac{1}{1+x} = 1 - x + x^2 + \cdots + (-1)^n x^n + o(x^n).
        \end{equation*}
          Or $x \mapsto \ln(1 + x)$ est une primitive de $\dps x \mapsto \frac{1}{1+x}$.
          \\
        Donc, le DL de $x \mapsto \ln(1 + x)$ est 
        \begin{equation*}
            \ln(1 + x) = x - \frac{x^2}{2} + \frac{x^3}{3} + \cdots + (-1)^n \frac{x^{n+1}}{n+1}. 
        \end{equation*}
         }
      \end{frame}
      %%%%
      \begin{frame}
\frametitle{Application}
        Ecrivons le développement limité à l'ordre $n$ de $\dps \frac{1}{1 + x^2}$ :

    \corrige{
    \begin{equation*}
        \frac{1}{1 + x^2} = 1 - x^2 + x^4 + (-1)^n x^{2n} + o(x^{2n}).
    \end{equation*}
    Or $x  \mapsto \dps  \frac{1}{1+x^2}$ est une primitive de $x \mapsto \arctan(x)$.
    \\
    Ainsi, le développement limité de $x \mapsto \arctan(x)$ est donné par 
    \begin{equation*}
        \arctan(x) = x - \frac{x^3}{3} + \frac{x^5}{5} + \cdots + (-1)^{2n+1}\frac{x^{2n+1}}{(2n+1)!}.
    \end{equation*}
    }
      \end{frame}
      %%
      \begin{frame}
        \frametitle{Recherche d'équivalents}
        \begin{proposition}

Si $f$ admet en $x_0$ un DL d'ordre $n$ dont la partie régulière est :
$\sum_{k=p}^{n} a_k (x - x_0)^k$ avec $a_p \neq 0$.
alors :
\begin{equation*}
  f(x) \underset{x_0}{\sim} a_p (x - x_0)^p.
  \end{equation*}
        \end{proposition}
        \invisible<1>{
    \textbf{Démonstration :}
     \begin{equation*}
        f(x) = \sum_{k=p}^{n} a_k (x - x_0)^k + o((x - x_0)^p)
    \end{equation*}
    et
    \begin{equation*}
        \frac{f(x)}{a_p(x-x_0)^p} = 1 + \frac{a_{p+1}}{a_p} (x - x_0) + \frac{a_{p+2}}{a_p} (x - x_0)^2 + \cdots + \frac{a_{n+p}}{a_p} (x - x_0)^n \alert{\underset{x_0}{\rightarrow} 1}.
    \end{equation*}
    \invisible<2>{
      }}
      \end{frame}
      %%%%
      \begin{frame}

        \frametitle{Exercice}
Déterminer un équivalent au voisinage de $0$ de la fonction $f$ définie sur $\mathbb{R}$ par
    \begin{equation*}
        f(x) = x \left( 1 + \cos(x) \right) - 2 \tan(x).
    \end{equation*}
    \corrige{:}
      \begin{enumerate}

      \item
        \invisible<1>{
\textcolor{midnightblue}{$f(-x) = -f(x) \ \alert{\Rightarrow}$  $f$ est impaire.
        La partie régulière du DL de $f$ ne contient que des puissances impaires de $x$.}

\item
  \invisible<2>{
\textcolor{midnightblue}{ DL en $0$ à l'ordre $3$ de $x \mapsto \cos(x)$
        \begin{equation*}
          \begin{split}
            \cos(x) &= 1 - \frac{x^2}{2} + o(x^{3}) \\
            \alert{\Rightarrow} & x \left(1 + \cos(x) \right) 
  = 2x - \frac{x^3}{2} + x o(x^3).
          \end{split}
        \end{equation*}
}
\invisible<3>{
  }}}
      \end{enumerate}
      \end{frame}
%%%%%%%%%%ùùù
      \begin{frame}
        \begin{enumerate}
          \setcounter{enumi}{2}
\invisible<1>{
        \item
          \textcolor{cadmiumgreen}{\textbf{Simplification des termes négligeables :}}
          \textcolor{midnightblue}{
          \begin{equation*}
            \begin{split}
              xo(x^3) &= x x^3 \varepsilon(x) \quad \text{avec} \quad \lim_{x \rightarrow 0} \varepsilon(x) = 0.
              \\
              & = x^4 \varepsilon(x) \quad \text{avec} \quad \lim_{x \rightarrow 0} \varepsilon(x) = 0.
              \\
              & = o(x^4).
              \end{split}
          \end{equation*}
          }
          \invisible<2>{
        \item
\textcolor{midnightblue}{\textcolor{cadmiumgreen}{\textbf{DL en $0$ à l'ordre $3$ de $x \mapsto x(1 + \cos(x))$}}
          \begin{equation*}
x \left(1 + \cos(x)  \right)= 2x - \frac{x^3}{2} + o(x^4).
          \end{equation*}
}
\invisible<3>{
\item
  \textcolor{midnightblue}{$\dps \tan(x) = \sin(x) \times \frac{1}{\cos(x)}$.
    Le DL de $x \mapsto \sin(x)$ à l'ordre $4$ est :
\begin{equation*}
        \sin(x) = x - \frac{x^3}{3!} + o(x^4).
\end{equation*}
  }
  \invisible<4>{
    }}}}
\end{enumerate}
      \end{frame}
      %%%%%%%%%%%%%%
 \begin{frame}
         \begin{enumerate}
          \setcounter{enumi}{5}
\invisible<1>{
        \item
          \textcolor{midnightblue}{
               \textbf{Transformation $\dps \frac{1}{u} \rightarrow \frac{1}{1 - u}$} 
\begin{equation*}
        \frac{1}{\cos(x)} = \frac{1}{1 - (1 - \cos(x))} = \frac{1}{1 - u(x)} \quad \text{avec} \quad u(x) = 1 - \cos(x)
\end{equation*}
          }
          \invisible<2>{
 \item
         \textcolor{midnightblue}{Or le DL à l'ordre $3$ au voisinage de $0$ de $\dps \frac{1}{1 - u}$ est :
$\frac{1}{1 - u} = 1 + u + u^2 + u^3 + o(u^3)$.
         }
         \invisible<3>{
        \item
          \textcolor{midnightblue}{Ainsi, le DL de $x \mapsto \dps \frac{1}{\cos(x)}$ est donné par
\begin{equation*}
    \begin{split}
        \frac{1}{\cos(x)} & = 1 + (1 - \cos(x)) + (1 - \cos(x))^2 + (1 - \cos(x))^3 + o((1 - \cos(x))^3)
        \\
        & = \frac{x^2}{2} - o(x^4) + \left(\frac{x^2}{2} - o(x^4)\right)^2
   + \left(\frac{x^2}{2} - o(x^4)\right)^3 + o \left( \left(\frac{x^2}{2} - o(x^4)\right)^3 \right).
        \end{split}
\end{equation*}
          }
          \invisible<4>{
            }}}}
\end{enumerate}
 \end{frame}
 %%%%
 \begin{frame}
   \begin{enumerate}
               \setcounter{enumi}{8}
\invisible<1>{
             \item
\textcolor{midnightblue}{DL d'un produit : on ne garde que les termes de degré $\leq 3$.
  \vspace*{0.3 cm}
  \begin{align*}
    \dps (1 - \cos(x))^2  &= - x^2o(x^3) + (o(x^3))^2 = -o(x^5) + o(x^6) = o(x^5). \\
    \\
\left( 1 - \cos(x) \right)^3 &= o(x^7).
  \end{align*}
}
\invisible<2>{
          \item
   \textcolor{midnightblue}{Simplification des termes négligeables,
    \begin{equation*}
o \left( \left(\frac{x^2}{2} - o(x^3)\right)^3 \right) = o(x^7).
    \end{equation*}
   }
   \invisible<3>{
         }}}
   \end{enumerate}
       \end{frame}
      %%%
      \begin{frame}
        \begin{enumerate}
                    \setcounter{enumi}{10}
         \invisible<1>{
          \item
\textcolor{midnightblue}{\textbf{DL de $\dps x \mapsto \frac{1}{\cos(x)}$ à l'ordre 3}
\begin{equation*}
    \frac{1}{\cos(x)}  = 1 + \frac{x^2}{2} + o(x^4).
\end{equation*}
}
\invisible<2>{
          \item
\textcolor{midnightblue}{\textbf{On obtient alors le DL à l'ordre $3$ de $x \mapsto \tan(x)$}
\begin{equation*}
    \tan(x) = \underbrace{\left( x - \frac{x^3}{3!} + o(x^4) \right)}_{\textcolor{cadmiumgreen}{\text{DL sin}}} \underbrace{\left( 1 + \frac{x^2}{2} + o(x^4) \right)}_{\textcolor{cadmiumgreen}{\text{DL 1/cos}}}
\end{equation*}
  \begin{equation*}
              \tan(x) = x - \frac{x^3}{3} + o(x^4).
  \end{equation*}
  }
     \invisible<3>{
         }}}
        \end{enumerate}
        \end{frame}
        %%%%
        \begin{frame}
  \begin{enumerate}
            \setcounter{enumi}{12}
\invisible<1>{
          \item
\textcolor{midnightblue}{DL au voisinage de $0$ de $f$ :
\begin{equation*}
  \begin{split}
  \alert{\text{DL} \ f(x)} & =  \alert{\text{DL} \left\{x(1 + \cos(x))  \right\} + \text{DL} \left\{-2 \tan(x) \right\} }  \\
  & = \left(2x - \frac{x^3}{2} + o(x^4)\right) - 2 \left(x - \frac{x^3}{3} + o(x^4)  \right)
  \\
  &= -\frac{7 x^3}{6} + o(x^4)
\end{split}
\end{equation*}
}
\invisible<2>{
\item
\textcolor{midnightblue}{\textbf{Equivalent de $f$ en $0$ :}
  \begin{equation*}
f(x) \underset{0}{\sim} -\frac{7}{6}x^3.
  \end{equation*}
}
\invisible<3>{
  }}}
\end{enumerate}
         \end{frame}
        %%%%%%
        \begin{frame}
          \frametitle{Etude de tangentes}
\invisible<1>{
          \textcolor{cadmiumgreen}{\textbf{DL d'ordre 1 :}}
 $f$ est dérivable en $x_0$ ssi $f$ admet un DL à l'ordre $1$ en $x_0$. Alors $f$ possède une tangente $T$ en $x_0$. La position de $\mathcal{C}_f$ par rapport à $T$ est donnée par le signe de
 \begin{equation*}
     f(x) - f(x_0) - (x - x_0)f^{\prime}(x_0)
 \end{equation*}
 \\
 \vspace*{0.3 cm}
 \invisible<2>{
 \textcolor{cadmiumgreen}{\textbf{DL d'ordre 2 :}}
 Si $f$ possède en $x_0$ un DL d'ordre $2$. Alors
 \begin{equation*}
     f(x) = a_0 + (x - x_0)a_1 + (x - x_0)^2 a_2 + o((x - x_0)^2) \quad \text{avec} \quad a_2 \neq 0.
 \end{equation*}
 \\
 \vspace*{0.3 cm}
 Alors la tangente est la droite d'équation $T_y = a_0 + a_1 (x - x_0)$ et, au voisinage de $x_0$, la position de $\mathcal{C}_f$ par rapport à $T_y$ est donnée par le signe de $a_2$, car : 
 \begin{equation*}
   \begin{split}
     f(x) - \left(  a_0 + a_1 (x - x_0)\right) & = (x - x_0)^2 a_2 + o((x - x_0)^2)
     \\
     & \underset{x_0}{\sim} a_2 (x - x_0)^2.
   \end{split}
 \end{equation*}
 \invisible<3>{
   }}}
   \end{frame}
   %%%
   \begin{frame}
 \textcolor{cadmiumgreen}{\textbf{DL d'ordre p :}} Si $f$ possède en $x_0$ un DL à un ordre $p \geq 2$ :
 \begin{equation*}
     f(x) = a_0 + a_1 (x - x_0) + a_2 (x - x_0)^2 + \cdots + a_p (x - x_0)^p + o((x - x_0)^p) \quad \text{avec} \quad a_p \neq 0
 \end{equation*}
On note $k$ le degré du premier coefficient non nul dans le DL à partir du degré $2$ et on note $a_k$ son coefficient.
\vspace*{0.2 cm}
\begin{itemize}
\item
  Si $k$ est pair et  $a_k > 0$ alors la courbe est au dessus de sa tangente.
  \vspace*{0.2 cm}
\item
  Si $k$ est pair et $a_k < 0$ alors la courbe est en dessous de sa tangente.
  \vspace*{0.2 cm}
\item
  Si $k$ est impair et $a_k > 0$ alors la courbe traverse sa tangente en passant au dessus.
  \vspace*{0.2 cm}
\item
  Si $k$ est impair et $a_k > 0$ alors la courbe traverse sa tangente en passant en dessous.
  \end{itemize}
        \end{frame}
        %%%
         \begin{frame}
     \frametitle{Application}
        $f : \mathbb{R} \rightarrow \mathbb{R}$  définie par $f(x) = \frac{1}{1 + e^x}$.
     Déterminer la position de la tangente à $\mathcal{C}_f$  en $0$.
     \\
       \vspace*{0.3 cm}
       \corrige{
         \begin{enumerate}
           \item \textcolor{cadmiumgreen}{\textbf{Transformation en DL usuel}}
               \begin{equation*}
          f(x) = \frac{1}{2 + (e^x - 1)} = \frac{1}{2}\frac{1}{\left( 1 + u(x) \right)} \quad \text{avec} \quad u(x) = \frac{e^x - 1}{2}.
               \end{equation*}
             \item
      \textcolor{cadmiumgreen}{\textbf{DL de $u \mapsto (1 + u)^{-1}$ en 0 à l'ordre $3$ en $0$}}          
      \begin{equation*}
          (1 + u)^{-1} = 1 - u + u^2 - u^3 + o(u^3).
      \end{equation*}
              \item
      \textcolor{cadmiumgreen}{\textbf{DL de $x \mapsto e^x$ en 0 à l'ordre $3$ en $0$}}
      \begin{equation*}
          e^x = 1 + x + \frac{x^2}{2} + \frac{x^3}{6} + o(x^3).
      \end{equation*}
         \end{enumerate}
         }
       \end{frame}
         %%%
         \begin{frame}
           Alors, $\dps u(x) = \frac{e^x - 1}{2} = \frac{x}{2} + \frac{x^2}{4} + \frac{x^3}{12} + o(x^3)$.
           \vspace*{0.2 cm}
      \begin{enumerate}
\setcounter{enumi}{3}
      \item
      \textcolor{cadmiumgreen}{\textbf{Règle du DL d'un produit :}}
      \begin{equation*}
              u^2(x) = \frac{x^2}{4} + \frac{x^3}{4} +  o(x^3)  
              \quad \text{et} \quad
              u^3(x)  = \frac{x^3}{8} + o(x^3)
      \end{equation*}
      \item
\textcolor{cadmiumgreen}{\textbf{DL de $f$ en $0$ :}} 
      \begin{equation*}
          f(x)  =\frac{1}{2} - \frac{1}{4}x + \frac{1}{48}x^3 + o(x^3)
      \end{equation*}
\item \textcolor{cadmiumgreen}{\textbf{Equation de la tangente à $f$ au point $0$ :}}
      \begin{equation*}
      g(x) = -\frac{1}{4}x + \frac{1}{2}    
      \end{equation*}
    \item
      \textcolor{cadmiumgreen}{\textbf{Signe de $f-g$ :}}
       \begin{equation*}
          f(x) - g(x) = \frac{1}{48}x^3 + o(x^3) \underset{0}{\sim} \frac{1}{48} x^3 > 0 \quad \text{pour x > 0}
      \end{equation*}

      \end{enumerate}

         \end{frame}
         %%%%
         \begin{frame}
      \frametitle{Illustration graphique}
       \begin{figure}[H]
           \centering
 \includegraphics[scale = 0.55]{Images/DL_tangente.pdf}
               \end{figure}

           \end{frame}
