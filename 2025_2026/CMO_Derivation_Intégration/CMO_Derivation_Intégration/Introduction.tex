 
%------------------------------------------------
\begin{frame}
	\frametitle{Objectifs}
            \begin{enumerate}
                \item 
Comprendre les comportements locaux et asymptotiques des fonctions
                \vspace{0.4 cm}
                \item
Savoir manipuler les développements limités
%% \vspace{0.4 cm}
%% \item
%% Connaître les principales propriétés des fractions rationnelles
\vspace{0.4 cm}
\item
  Savoir calculer plusieurs familles d'intégrales
  \vspace{0.4 cm}
\item
Introduction à l'analyse à plusieurs variables.
\end{enumerate}   
\end{frame}

\begin{frame}
    \frametitle{Contenu du module}
    \begin{enumerate}
        
   \item
            Chapitre 1 : Relations de comparaison \quad \alert{(CMO 1)}
                       \vspace{0.15 cm}
\begin{itemize}
\item
Un peu de topologie,  
       continuité d’une fonction en un point.
    \item 
    Fonctions dominées, fonctions négligeables, fonctions équivalentes.
\end{itemize}
   \vspace{0.3 cm}
\item
    Chapitre 2 : Développements limités \quad \alert{(CMO $2$)}
               \vspace{0.15 cm}

\begin{itemize}
    \item 
    Formules de Taylor, opérations sur les développements limités, applications.
    \item \textcolor{cadmiumgreen}{\textbf{Contrôle continu 1h 14 Mars 2023}}
\end{itemize}
\vspace{0.15 cm}
\item
Chapitre 3 : Calcul d'intégrales \quad \alert{(CMO 3)}
\vspace{0.15 cm}
\item
Chapitre 4 : Introduction à l'analyse à plusieurs variables \quad \alert{(CMO 4)}
\end{enumerate}            
        
\end{frame}
